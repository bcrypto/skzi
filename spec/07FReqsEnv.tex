\chapter{Функциональные требования безопасности к среде}\label{FReqsEnv}

\section{Требования по идентификации и аутентификации (ИА)}

\req{ИА}{1, 2}\label{Identification}
Каждому оператору должен быть назначен идентификатор и набор 
ролей~\useref{Roles}.

\req{ИА}{1, 2}\label{AuthData}
Для каждого идентификатора оператора~\useref{Identification}
должны быть определены аутентификационные данные. 
Среди аутентификационных данных должны быть выделены 
секреты аутентификации.
%
Устройства ввода аутентификационных данных 
должны быть включены в список критических системных 
компонентов~\forref{CSCList}.

\req{ИА}{1, 2}\label{Authentication}
Должны быть определены и корректно реализованы средства 
аутентификации для проверки 
подлинности идентификатора оператора и возможности выполнения 
оператором сервисов соответствующих ролей~\useref{DAC}.
%
Средства аутентификации, реализуемые~\TOE, 
должны быть включены в список сервисов~\forref{Services}.

\req{ИА}{2}\label{Auth2Factor}
Должно использоваться не менее двух факторов аутентификации.

\req{ИА}{1, 2}\label{AuthStrength}
Вероятность пройти аутентификацию, 
не зная секретов аутентификации, 
не должна превышать~$10^{-6}$, 
если предпринимается одна попытка аутентификации, 
и не должна превышать $10^{-5}$, 
если предпринимаются попытки в течение $1$~мин.

\req{ИА}{1, 2}\label{PwdMask}
Информация, которая отображается при вводе секретов аутентификации, 
не должна ослаблять стойкость средств аутентификации.

\req{ИА}{1, 2}\label{PwdSet}
Должны быть определены и реализованы средства проверки качества 
секретов аутентификации. 
Средства должны применяться при каждой установке или смене секрета.

\req{ИА}{1, 2}\label{AuthSecrets}
При реализации средств аутентификации в~\TOE
контрольные значения аутентификационных данных 
должны быть отнесены к открытым объектам~\forref{Objects}.
Контрольные значения секретов аутентификации
должны быть отнесены к критическим объектам~\forref{Objects}.
Сеансовые объекты, которые содержат значения секретов аутентификации,
также должны быть отнесены к критическим объектам~\forref{Objects}.

\section{Требования по настройке среды (НС)}

\req{НС}{1, 2}\label{ENVInstall}
Должны быть определены настройки среды эксплуатации
для безопасной установки~\TOE уполномоченным администратором.

\req{НС}{1, 2}\label{ENVObjects}
Должны быть определены настройки среды эксплуатации
для обеспечения целостности долговременных объектов~\useref{Objects} 
при их хранении внутри криптографической границы. 
Выбранные настройки должны предотвращать 
изменение долговременных объектов вне сеансов между операторами и~\TOE.

\req{НС}{1, 2}\label{ENVSession}
Должны быть определены настройки среды эксплуатации
для обеспечения конфиденциальности и целостности 
сеансовых объектов~\useref{Objects}  
и аутентификационных данных при их передаче и обработке 
в критических системных компонентах во время сеансов операторов.


