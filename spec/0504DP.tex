% Data Protection
\section{Защита объектов (ЗО)}\label{DP}

\subsection{Обзор}\label{DP.Defs}

Пакет ЗО устанавливает требования по защите объектов СКЗИ. 
Защита устанавливается или снимается при записи и чтении объектов 
во время их хранения в пределах криптографической границы, экспорте за ее 
пределы и импорте из-за пределов. 

Предусмотрены следующие методы защиты.

1~{\it Криптографические методы}. 
Состоят в применении алгоритмов шифрования для обеспечения конфиденциальности, 
алгоритмов ЭЦП и имитозащиты для контроля целостности и подлинности.

2~{\it Аппаратные методы}. 
Состоят в аппаратной защите от несанкционированного чтения и (или) модификации
областей памяти, в которых размещаются целевые объекты. 
%
Защита может состоять в размыкании цепей памяти при попытках 
несанкционированного доступа, использовании защищенных модулей памяти, 
контроле четности на аппаратном уровне, хранении объектов в гарантированно 
неизменяемых областях памяти.
%
Аппаратная защита обеспечивается применением смарт-карт, токенов и других
подобных устройств. В их качестве может выступать отдельный аппаратный СКЗИ.
%
Аппаратная защита СКЗИ распространяется на объекты, которые хранятся 
(даже в \addendum{незашифрованном} виде) в пределах его криптографической 
границы~\forref{R.DP.Hard}.

3~{\it Методы разделения секрета}. 
Состоят в разбиении защищаемого критического объекта на частичные секреты, 
каждый из которых затем защищается по отдельности.
%
Простейшим методом разделения секрета является представление
ключа как двоичного слова в виде суммы (поразрядной по модулю $2$)
нескольких частичных секретов. 
В этом случае для восстановления исходного 
ключа требуется располагать всеми частичными секретами.
%
Более сложные пороговые методы позволяют 
определить исходный ключ при наличии не всех, 
а только порогового числа частичных секретов, 
например трех из пяти~\forref{R.DP.Split}.

4~{\it Алгоритмические методы}. 
Состоят в контроле целостности с помощью некриптографических
алгоритмов, бесключевых криптографических алгоритмов
или криптографических алгоритмов с известными всем ключами.
%
К алгоритмическим методам относятся: 
сверка нескольких копий объекта, 
проверка контрольных хэш-значений,
хранение объекта вместе с результатом его зашифрования на нулевом ключе,
проверка самоподписанного сертификата открытого ключа~\forref{R.DP.Algo}.

5~{\it Организационные меры}. 
Состоят в обеспечении конфиденциальности с помощью мероприятий,
не относящихся к информационным технологиям.

При выборе методов защиты предпочтение следует отдавать криптографическим. 
Однако для организации криптографической защиты объектов требуется использовать 
другие объекты-ключи. Эти объекты также нужно защищать, следовательно, нужно 
использоваться новые ключи и т.~д. Аппаратные методы и методы разделения 
секрета позволяют прервать цепочку ключей защиты других ключей.

В алгоритмических методах защиты контрольная характеристика, 
на основании которой принимается решение о целостности объекта, 
может быть вычислена любой стороной. Поэтому алгоритмический метод позволяет 
контролировать только целостность объекта и защищать его только от случайных 
сбоев в системной среде, но не от преднамеренного воздействия. 
%
Однако если контрольная характеристика защищается, то защита распространяется
на контролируемый объект. 
%
Пусть, например, СКЗИ вырабатывает личный и открытый ключи,
сохраняет личный ключ на смарт-карту, 
а открытый ключ отсылает в удостоверяющий центр для получения сертификата. 
%
После получения сертификата СКЗИ вычисляет по сохраненному личному ключу 
открытый и сравнивает его с ключом, размещенным в сертификате.
%
Проверка связи между ключами соответствует алгоритмическому методу защиты.
Аппаратная защита личного ключа распространяется в момент импорта
на открытый ключ сертификата~\forref{R.DP.Algo}.

Защита экспортируемого критического объекта выполняется непосредственно в момент
экспорта. Для сравнения, защита экспортируемого открытого объекта может быть
отложена: выполнена позже силами доверенных СКЗИ в системной среде.
%
Установка защиты в любом случае будет проверена при импорте. 
%
Данные правила защиты распространяются на запись и чтение объектов при 
их хранении в пределах криптографической границы~\forref{R.DP.Export,R.DP.Import}. 

Экспортом не считается передача объекта во владение другому СКЗИ в системной
среде.
%
Например, передача случайных чисел от средства их генерации средству
персонализации аппаратных криптографических устройств, которое строит по 
случайным числам личные ключи устройств и записывает их на устройства
(экспорт)~\forref{R.DP.Export}. 

Защита критических объектов предполагает их надежную очистку.
%
Очистка объектов в энергозависимой памяти может быть произведена перезаписью 
ячеек памяти либо отключением питания.
%
Для очистки содержимого энергонезависимой памяти следует использовать
многократную перезапись ячеек памяти битовыми строками со средним, малым и 
большим числом единиц. Рекомендуется проводить несколько повторов трехцикловой 
перезаписи сначала случайной строкой, затем нулевой строкой и 
наконец строкой из всех единиц. 
%
После циклов перезаписи рекомендуется проводить контрольное считывание ячеек 
памяти~\forref{R.DP.Zero}.

\subsection{Требования}\label{DP.Reqs}

\req{ЗО}{1--4}\label{R.DP.Crit}
Должна обеспечиваться конфиденциальность критических объектов~\useref{R.AC.Objects}.
%
\addendum{Используемые для этого методы защиты должны выбираться} 
из следующего списка: криптографические, аппаратные, разделения секрета.
%
При защите частичных секретов список дополняется организационными мерами.

\req{ЗО}{1--4}\label{R.DP.Public}
Должен осуществляться контроль целостности и подлинности критических и открытых
объектов~\useref{R.AC.Objects}. 
%
\addendum{Используемые для этого методы защиты должны выбираться} 
из следующего списка: криптографические, аппаратные, алгоритмические.

\req{ЗО}{1--4}\label{R.DP.CryptoE}
Должны быть определены и \addendum{корректно 
реализованы~\forref{R.CR.Embed,R.CR.Selected,R.CR.Detailed}}
криптографические методы обеспечения конфиденциальности.
%
Методы должны быть основаны на алгоритмах шифрования~\useref{R.CS.Algs}.
%
Личные и секретные ключи алгоритмов должны быть~\addendum{\forref{R.AC.Objects}}
отнесены к критическим объектам, а открытые ключи и долговременные 
параметры~--- \addendum{\forref{R.AC.Objects}}~к открытым объектам. 

\begin{note}
При планировании защиты следует руководствоваться правилом:
ключ защиты не должен быть слабее защищаемого критического объекта. 
%
Например, не следует устанавливать защиту 256-битового ключа на 128-битовом. 
\end{note}

\req{ЗО}{1--4}\label{R.DP.CryptoI}
Должны быть определены и \addendum{корректно 
реализованы~\forref{R.CR.Embed,R.CR.Selected,R.CR.Detailed}} 
криптографические методы контроля
целостности и подлинности. Методы должны быть основаны на алгоритмах 
ЭЦП и имитозащиты~\useref{R.CS.Algs}.
%
Личный ключ ЭЦП и секретный ключ имитозащиты должны быть~\forref{R.AC.Objects} 
отнесены к критическим объектам, а открытый ключ ЭЦП и долговременные 
параметры~--- \addendum{\forref{R.AC.Objects}}~к открытым объектам. 
%
Длина имитовставки в битах должна быть не меньше~$64$.

\req{ЗО}{1--4}\label{R.DP.Hard}
Должны быть определены и корректно использованы аппаратные методы защиты.
Устройства, которые реализуют аппаратные методы, 
должны соответствовать ТНПА в части физической безопасности.
%
При отсутствии подходящих ТНПА должен быть проведена оценка надежности 
устройств.

\begin{note}
При оценке надежности следует учитывать требования пакета ФБ (\ref{PS}),
декларации разработчиков устройств, сведения о применении устройств, опыт 
применения.
\end{note}

\req{ЗО}{1--4}\label{R.DP.Split}
Должны быть определены и \addendum{корректно 
реализованы~\forref{R.CR.Embed,R.CR.Selected,R.CR.Detailed}} 
методы разделения секрета. При восстановлении критического объекта должно 
использоваться не менее двух различных частичных секретов.
%
Если для восстановления критического объекта требуется $k$ частичных секретов,
то любые $k-1$ частичных секретов не должны давать никакой информации об 
исходном объекте.
%
Частичные секреты должны быть~\forref{R.AC.Objects} отнесены к критическим объектам.

\begin{note}
\addendum{
Владельцы частичных секретов не обязательно различны. 
Один оператор может владеть сразу всеми частичными секретами.
}
\end{note}

\req{ЗО}{1--4}\label{R.DP.Algo}
Должны быть определены и \addendum{корректно
реализованы~\forref{R.CR.Selected,R.CR.Detailed}} алгоритмические методы
контроля целостности.
%
Алгоритмические методы контроля должны гарантировать, \addendum{что}
\begin{itemize}
\item
вероятность необнаружения случайной модификации контролируемого 
объекта при его хранении не превышает~$2^{-32}$;
\item
вероятность необнаружения намеренной модификации контролируемого 
объекта при его импорте не превышает~$2^{-128}$.
\end{itemize}
%
Если контролируемый объект не является частичным секретом или системным
объектом, то его контрольная характеристика должна быть~\forref{R.AC.Objects}
отнесена к открытым или критическим объектам.

\req{ЗО}{1--4}\label{R.DP.Org}
Должны быть~\forref{R.GD.Admin,R.GD.Roles} определены и изложены в руководствах
организационные меры по обеспечению конфиденциальности частичных секретов. Меры
должны быть направлены на ограничение физического доступа к носителям
информации, на которых хранятся секреты, и \addendum{противодействие}
попаданию порогового числа частичных секретов \addendum{противнику}.

\req{ЗО}{1--4}\label{R.DP.Export}
При записи на хранение и экспорте критических объектов должна 
устанавливаться их защита~\useref{R.DP.Crit}.

\begin{note}
Экспорт и импорт могут проводиться в рамках онлайн-взаимодействия с удаленным
оператором. При использовании криптографических методов ключи защиты 
могут генерироваться с помощью протоколов формирования общего 
ключа~\useref{R.CS.Gen}.
\end{note}

\req{ЗО}{1--4}\label{R.DP.Import}
При чтении во время хранения и импорте критических и открытых объектов  
должен проводиться контроль их целостности и подлинности~\useref{R.DP.Public}. 
При ошибке контроля использование объекта должно быть запрещено. 

\req{ЗО}{1--4}\label{R.DP.System}
Контроль целостности системных объектов~\useref{R.DP.Public} 
должен~\forref{R.ST.Tests} проводиться при самотестировании.

\req{ЗО}{1--4}\label{R.DP.Session}
Все сеансовые критические объекты~\useref{R.AC.Objects}
должны очищаться до завершения сеансов.

\req{ЗО}{1--4}\label{R.DP.Zero}
Очистка долговременных и сеансовых критических объектов~\useref{R.AC.Objects} 
должна быть реализована так, чтобы после очистки нельзя было определить 
первоначальное значение объекта. 
