% Tests
\section{Программа испытаний (ПИ)}\label{TE}

\subsection{Обзор}\label{TE.Intro}

Пакет ТЕ устанавливает требования к тестированию СКЗИ.
%
Тесты должны продемонстрировать корректную работу СКЗИ, 
соответствие функциональной спецификации.
%
Качество тестирования повышается, если оно покрывает не только внешние 
интерфейсы, но и внутренние, т.~е. проверяется корректность 
внутренних компонентов СКЗИ.

\subsection{Требования}\label{TE.Reqs}

\req{ПИ}{1--4}\label{R.TE.Prg}
Должна быть разработана программа испытаний~СКЗИ разработчиком.
Программа должна определять:
\begin{itemize}
\item[--]
планы тестирования;
\item[--]
содержание тестов;
\item[--]
ожидаемые результаты выполнения тестов;
\item[--]
фактические результаты выполнения тестов.
\end{itemize}

\req{ПИ}{1--4}\label{R.TE.Coverage}
Тесты программы испытаний~\useref{R.TE.Prg} должны 
покрывать все функциональные возможности~СКЗИ, 
определенные в функциональной спецификации.

\req{ПИ}{2--4}\label{R.TE.Deep}
Тесты программы испытаний~\useref{R.TE.Prg} должны 
покрывать функциональные возможности всех компонентов~СКЗИ, 
определенных в описании программ~\useref{R.DI.ProgSpec}.

