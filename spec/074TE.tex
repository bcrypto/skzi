% Tests
\section{Программа испытаний (ПИ)}\label{TE}

\subsection{Обзор}\label{TE.Intro}

Пакет ПИ устанавливает требования к тестированию СКЗИ.
%
Тесты проверяют корректную работу СКЗИ, 
соответствие функциональной спецификации,
описаниям программ и аппаратных компонентов.

Качество тестирования повышается, если оно покрывает не только внешние 
интерфейсы~\useref{R.DI.HLD}, но и внутренние, 
т.~е. если проверяется корректность внутренних компонентов 
СКЗИ~\useref{R.DI.LLD}. 

\subsection{Требования}\label{TE.Reqs}

\req{ПИ}{1--4}\label{R.TE.Prg} % 1
Должна быть разработана программа испытаний~СКЗИ разработчиком.
Программа должна определять:
\begin{itemize}
\item[--]
планы тестирования;
\item[--]
содержание тестов;
\item[--]
ожидаемые результаты выполнения тестов;
\item[--]
фактические результаты выполнения тестов.
\end{itemize}

\req{ПИ}{1--4}\label{R.TE.Coverage} % 2
Тесты программы испытаний~\useref{R.TE.Prg} должны 
покрывать все функциональные возможности~СКЗИ, 
определенные в функциональной спецификации.

\req{ПИ}{2--4}\label{R.TE.Deep} % 3
Тесты программы испытаний~\useref{R.TE.Prg} должны 
покрывать функциональные возможности всех компонентов~СКЗИ, 
определенных в описаниях~\useref{R.DI.Spec}.

