\section{Назначение}\label{COMMON.Purpose}

Настоящий стандарт устанавливает требования безопасности к средствам
криптографической защиты информации. Требования предназначены для использования
потребителями (заказчиками) при выборе и применении СКЗИ,
разработчиками при создании СКЗИ и
экспертами при оценке надежности СКЗИ.

СКЗИ реализует один или несколько криптографических сервисов, с помощью 
которых операторы средства решают различные задачи защиты информации.
%
Для обеспечения безопасной работы сервисов в СКЗИ реализуются 
дополнительные функции и механизмы: управления ключами,
контроля доступа, проверки работоспособности, аудита и др.

Требования безопасности делятся на функциональные и гарантийные.
%
Функциональные требования отвечают за противодействие угрозам безопасности, 
следование определенным правилам безопасности.
%
Гарантийные требования обеспечивают доверие к тому, что СКЗИ корректно
спроектировано и разработано, протестировано в достаточном объеме, правильно
установлено и эксплуатируется.

СКЗИ использует ресурсы системной среды, безопасность средства зависит от 
работоспособности и надежности компонентов среды. 
%
Поэтому функциональные требования выдвигаются как к СКЗИ, 
так и к его системной среде. 

Если системные компоненты не входят в состав СКЗИ, как в случае программного 
средства, то требования к среде адресуются потребителю СКЗИ. 
Потребитель настраивает среду для безопасного применения средства. 
%
Однако если некоторый системный компонент включен в состав СКЗИ, то требования, 
его касающиеся, реализуются самим СКЗИ и, таким образом, адресуются 
разработчику.

Требования безопасности схожего назначения группируются в пакеты. 
Перечень пакетов приводится в таблице~\ref{Table.COMMON.Packages}.

Пакет может быть обязательным или необязательным. Решение о применении 
необязательного пакета принимается разработчиком и (или) потребителем 
с учетом функционального назначения СКЗИ и условий его эксплуатации.

\begin{table}[hbt]
\caption{Пакеты}\label{Table.COMMON.Packages}
\begin{tabular}{|l|c|c|c|c|}
\hline
\multicolumn{1}{|c|}{\multirow{2}{*}{Пакет}} & 
\multirow{2}{*}{Код} & 
\multirow{2}{*}{Обязательность} & 
Начальный & Раздел\\
&&& уровень & стандарта\\
\hline
\hline
\multicolumn{5}{|c|}{функциональные требования к средству}\\
\hline
Криптографическая поддержка & КП & $+$ & 1 & \ref{CS}\\
Реализация сервисов         & РС & $+$ & 1 & \ref{SV}\\
Управление доступом         & УД & $+$ & 1 & \ref{AC}\\
Защита объектов             & ЗО & $+$ & 1 & \ref{DP}\\
Самотестирование            & СТ & $+$ & 1 & \ref{ST}\\
Аудит                       & АУ & $+$ & 3 & \ref{AU}\\
Физическая безопасность     & ФБ & $+$ & 3 & \ref{PS}\\
Защита от воздействий       & ЗВ & $+$ & 3 & \ref{EF}\\
Защита от утечек            & ЗУ & $+$ & 4 & \ref{NI}\\
Генерация случайных чисел   & СЧ & $-$ & 1 & \ref{RN}\\
Обновление программ         & ОП & $-$ & 1 & \ref{SU}\\
Вывод из эксплуатации       & ВЭ & $-$ & 1 & \ref{DE}\\
\hline
\multicolumn{5}{|c|}{функциональные требования к системной среде}\\
\hline
Идентификация и аутентификация & ИА & $+$ & 1 & \ref{IA}\\
Настройка среды                & НС & $+$ & 1 & \ref{ES}\\
Доверенный канал               & ДК & $-$ & 1 & \ref{TC}\\
\hline
\multicolumn{5}{|c|}{гарантийные требования}\\
\hline
Проектирование и разработка & ПР & $+$ & 1 &\ref{DI}\\
Поддержка жизненного цикла  & ЖЦ & $+$ & 1 &\ref{LC}\\
Руководства                 & РД & $+$ & 1 &\ref{GD}\\
Программа испытаний         & ПИ & $+$ & 1 &\ref{TE}\\
Анализ программ             & АП & $+$ & 1 &\ref{CR}\\
\hline
\end{tabular}
\end{table}

Требования безопасности обязательных пакетов группируются в наборы, называемые 
уровнями безопасности. 
%
Определены 4 уровня: 1~(базовый), 2~(средний), 3~(высокий), 4~(максимальный).
%
Каждый следующий уровень включает требования предыдущего и, таким образом,
с увеличением уровня требования усиливаются.
%
Направления усиления в рамках общей концепции требований представлены 
в~\ref{COMMON.Concept}.

\begin{note}
Имеется практика деления СКЗИ на программные и программно-аппаратные не по
исполнению, как в настоящем стандарте, а по гарантиям безопасности. 
Согласно этой практике, программно-аппаратные средства считаются более 
защищенными, именно их требуется использовать для защиты служебной информации 
ограниченного распространения.
%
Настоящий стандарт вводит явное деление по уровням безопасности, при котором 
для защиты служебной информации следует использовать СКЗИ уровней 3 и 4.
\end{note}

СКЗИ определенного уровня должно удовлетворять всем требованиям этого уровня
и может удовлетворять дополнительным требованиям.
%
В частности, уровень может дополняться требованиями необязательных пакетов.

\begin{note}
Требование может быть привязано к условию, которое выполняется или не выполняется 
в зависимости от специфики СКЗИ, или может относиться к специфическому для СКЗИ 
списку объектов, методов, ролей и др. 
%
Такое требование является условно обязательным. Оно считается автоматически 
выполненным, если условие не соблюдено или список пуст. 
\end{note}

Если уровень усиливается необязательными пакетами, то коды этих пакетов 
перечисляются в квадратных скобках после номера уровня и знака плюс. 
В результате получается код усиленного уровня, который можно использовать 
для характеризации СКЗИ. Пример: 2+[ОП,ДК].

Решение об использовании СКЗИ того или иного уровня безопасности принимает 
потребитель. В~\ref{COMMON.Levels} определены принципы формирования уровней, 
которые следует учитывать при принятии решения.

Решение о применении необязательных пакетов принимает разработчик СКЗИ с учетом 
функциональных особенностей средства и специфики условий его эксплуатации. 
%
Пакет СЧ следует применять тогда, когда в СКЗИ генерируются секретные и личные 
ключи, другие критические объекты.
%
Пакет ОП нужно использовать, если планируется обновление отдельных программных 
компонентов СКЗИ без полной переустановки программ.  
%
Пакет ВЭ обеспечивает надежную очистку объектов СКЗИ перед завершением 
эксплуатации средства.
%
Наконец, пакет ДК следует применять тогда, когда операторам СКЗИ 
предоставляется удаленный доступ к средству.

Способ реализации требований безопасности определяется
в функциональной спецификации СКЗИ. Это документ или набор документов,
рекомендуемое содержание которого представлено в приложении~\ref{SPEC}.
