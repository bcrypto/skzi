% Physical Security
\section{Физическая безопасность (ФБ)}\label{PS}

\subsection{Обзор}\label{PS.Intro}

Пакет ФБ устанавливает требования к механизмам защиты аппаратного СКЗИ 
от несанкционированного физического доступа к компонентам внутри 
криптографической границы и предотвращения несанкционированного использования 
или модификации СКЗИ.
%
Физическая безопасность СКЗИ предполагает использование надежных аппаратных 
КСК. 

\addendum{
В СКЗИ применяют пассивные и активные механизмы защиты.
}

\addendum{При применении пассивных механизмов защиты попытка доступа} 
приводит к появлению следов, которые можно обнаружить. Используются защитные 
покрытия, корпусирование, пломбирование, опечатывание и 
др.~\forref{R.PS.PassiveDetection}.

Защитное покрытие или корпус могут быть несъемными в том смысле, что попытка их 
снятия приводит к повреждению компонентов или электрического монтажа СКЗИ. 
%
\addendum{При этом обеспечивается защита от несанкционированного доступа, 
при котором противник стремится остаться незамеченным.}
%
Пример несъемного покрытия~--- заливка из эпоксидного 
материала~\forref{R.PS.ActiveDetectionSensor}. 
 
В СКЗИ может сохраняться возможность доступа к внутренним компонентам,
например, через съемные элементы корпуса. 
\addendum{
В таких случаях предусматриваются активные механизмы защиты. При их применении 
попытка доступа приводит к срабатыванию датчиков контроля доступа и 
автоматической реакции СКЗИ~\forref{R.PS.ActiveDetectionSensor,R.PS.Erasing}.}

\subsection{Требования}\label{PS.Reqs}

\req{ФБ}{3, 4}\label{R.PS.List} % 1
Должны быть определены и корректно реализованы механизмы физической защиты СКЗИ. 

\req{ФБ}{3, 4}\label{R.PS.Production} % 2
Аппаратные КСК должны быть изготовлены \addendum{в условиях серийного 
производства в соответствии с техническими стандартами, согласно технической 
документации (техническим условиям).}

\req{ФБ}{4}\label{R.PS.Passivation} % 3
Аппаратные КСК должны иметь покрытие, обеспечивающее защиту от 
\addendum{повреждений из-за} воздействия окружающей среды. 

\begin{note*}
\addendum{Может использоваться иммерсионное золочение или паяльная маска 
контактных площадок и концевых ламелей, слой диэлектрической защиты 
(лак, эмаль) печатных плат.}
\end{note*}

% корпус

\req{ФБ}{3, 4}\label{R.PS.Coating} % 4
СКЗИ должен быть окружен твердым непрозрачным корпусом и (или) покрытием.
%
Визуальный осмотр СКЗИ не должен давать дополнительную по отношению 
к конструкторской документации~\useref{R.DI.Spec} информацию 
\addendum{о К}СК, а также информацию об их текущем состоянии и выполняемых ими  
операциях.

% пассивное обнаружение

\req{ФБ}{3, 4}\label{R.PS.PassiveDetection} % 5
При попытке физического доступа к \addendum{компонентам внутри 
криптографической границы} СКЗИ механизмы физической защиты~\useref{R.PS.List} 
должны обеспечивать обнаружение доступа. 

\req{ФБ}{3, 4}\label{R.PS.PassiveDetectionId} % 6
Если в СКЗИ для обнаружения доступа к \addendum{компонентам внутри 
криптографической границы} предусмотрены печати, пломбы или иные подобные 
элементы, то 
\begin{itemize}
\item
\addendum{они должны} 
иметь уникальные номера или идентификаторы;
\item
\addendum{они должны}
сохранять конструкцию и внешний вид при эксплуатации СКЗИ в допустимых 
диапазонах \addendum{параметров эксплуатации}~\useref{R.EF.Ranges};
\item
\addendum{условия и порядок их установки и/или замены
должны быть определены~\forref{R.GD.Admin,R.GD.Roles} в руководствах}.
\end{itemize}

% активный контроль

\req{ФБ}{4}\label{R.PS.ActiveDetectionSensor} % 7
Механизмы физической защиты~\useref{R.PS.List} должны обеспечивать 
защиту от несанкционированного доступа к \addendum{компонентам внутри 
криптографической границы} СКЗИ. 
%
Должны использоваться \addendum{несъемный корпус (покрытие)}~\useref{R.PS.Coating} 
и (или) активные методы защиты с помощью \addendum{датчиков} контроля доступа.

\req{ФБ}{4}\label{R.PS.Erasing} % 8
При срабатывании \addendum{датчиков} контроля доступа СКЗИ должно 
переходить в состояние полной блокировки~\useref{R.AC.CrashState}.
%
\addendum{Переход должен выполняться как при включенном электропитании 
СКЗИ, так и отключенном.}

\req{ФБ}{4}\label{R.PS.SensorsQuality} % 9
Тип \addendum{датчиков} контроля доступа, их количество, места 
размещения, порядок функционирования должны выбираться таким образом, 
чтобы была исключена возможность доступа к \addendum{компонентам внутри 
криптографической границы} СКЗИ без срабатывания \addendum{датчиков}.

\req{ФБ}{4}\label{R.PS.SensorsFaults} % 10
Должна быть предусмотрена защита от непроизвольного срабатывания \addendum{датчиков} 
контроля доступа. \addendum{Оценка вероятности} непроизвольного 
срабатывания, а также \addendum{способ защиты от него} должны 
быть приведены~\forref{R.PS.List} в описании механизмов физической защиты СКЗИ. 

