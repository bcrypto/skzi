% Physical Security
\section{Физическая безопасность (ФБ)}\label{PS}

\subsection{Обзор}\label{PS.Intro}

Пакет ФБ устанавливает требования к механизмам защиты аппаратного СКЗИ 
от несанкционированного физического доступа к компонентам внутри 
криптографической границы и предотвращения несанкционированного использования 
или модификации СКЗИ.
%
Физическая безопасность СКЗИ предполагает использование надежных аппаратных 
КСК. 

В СКЗИ применяют средства обнаружения доступа к внутренним компонентам.
%
Используются защитные покрытия, корпусирование, пломбирование, опечатывание и др. 
%
Попытки доступа приводят к появлению следов, которые можно 
обнаружить~\forref{R.PS.PassiveDetection}. 

В СКЗИ может сохраняться возможность доступа к внутренним компонентам,
например, через съемные элементы корпуса. В таких случаях предусматриваются
технические средства контроля доступа, например, датчики вскрытия корпуса.
%
Технические средства являются активными в том смысле, что попытка доступа
приводит к реакции средств безопасности СКЗИ~\forref{R.PS.ActiveDetectionSensor}.

Технические средства контроля доступа необязательны при использовании пассивных
методов защиты~--- цельного корпуса или покрытия, попытка снятия которого 
приводит к повреждению внутренних компонентов или электрического монтажа СКЗИ. 
%
Пример несъемного покрытия~--- заливка из эпоксидного 
материала~\forref{R.PS.ActiveDetectionSensor}.

\subsection{Требования}\label{PS.Reqs}

\req{ФБ}{3, 4}\label{R.PS.List}
Должны быть определены и корректно реализованы механизмы физической защиты СКЗИ. 

\req{ФБ}{3, 4}\label{R.PS.Production}
Аппаратные КСК должны быть изготовлены промышленным способом: в соответствии с 
промышленными стандартами, согласно технической документации (техническим 
условиям).

\req{ФБ}{4}\label{R.PS.Passivation}
Аппаратные КСК должны иметь покрытие, обеспечивающее защиту от воздействия 
окружающей среды. 

\begin{note}
Может использоваться иммерсионное золочение или паяльная маска контактных 
площадок и концевых ламелей, слой диэлектрической защиты (лак, эмаль) 
печатных плат.
\end{note}

% корпус

\req{ФБ}{3, 4}\label{R.PS.Coating}
СКЗИ должен быть окружен твердым непрозрачным корпусом и (или) покрытием.
%
Визуальный осмотр СКЗИ не должен давать дополнительную по отношению 
к конструкторской документации~\useref{R.DI.HardSpec} информацию о внутренних 
КСК, а также информацию об их текущем состоянии и выполняемых ими 
операциях.

% пассивное обнаружение

\req{ФБ}{3, 4}\label{R.PS.PassiveDetection}
При попытке физического доступа к внутренним компонентам СКЗИ механизмы 
физической защиты~\useref{R.PS.List} должны обеспечивать обнаружение доступа. 

\req{ФБ}{3, 4}\label{R.PS.PassiveDetectionId}
Если в СКЗИ для обнаружения доступа к внутренним компонентам предусмотрены 
печати, пломбы или иные подобные элементы, то они должны:
\begin{itemize}
\item
иметь уникальные номера или идентификаторы;
\item
сохранять конструкцию и внешний вид при эксплуатации СКЗИ в допустимых 
диапазонах условий окружающей среды и рабочих параметров~\useref{R.EF.Ranges}. 
\end{itemize}

% активный контроль

\req{ФБ}{4}\label{R.PS.ActiveDetectionSensor}
Механизмы физической защиты~\useref{R.PS.List} должны обеспечивать 
защиту от несанкционированного доступа к внутренним компонентам
СКЗИ. 
%
Должны использоваться либо пассивные методы защиты, состоящие в применении 
несъемного корпуса и (или) покрытия~\useref{R.PS.Coating}, либо 
активные методы защиты с помощью технических средств контроля доступа.

\req{ФБ}{4}\label{R.PS.Erasing}
При срабатывании технических средств контроля доступа СКЗИ должно 
переходить в состояние полной блокировки~\useref{R.AC.CrashState}.

\req{ФБ}{4}\label{R.PS.SensorsQuality}
Тип технических средств контроля доступа, их количество, места 
размещения, порядок функционирования должны выбираться таким образом, 
чтобы была исключена возможность доступа к внутренним компонентам СКЗИ 
без срабатывания средств контроля доступа.

\req{ФБ}{4}\label{R.PS.SensorsFaults}
Должна быть предусмотрена защита от непроизвольного срабатывания технических 
средств контроля доступа. Способ защиты от непроизвольного срабатывания, 
а также оценка ее вероятности должны быть~\forref{R.PS.List} 
приведены в описании механизмов физической защиты СКЗИ.
