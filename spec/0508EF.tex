% Environmental Failure protection
\section{Защита от воздействий (ЗВ)}\label{EF}

\subsection{Обзор}\label{EF.Intro}

Пакет ЗВ определяет требования \addendum{по защите} аппаратного СКЗИ при выходе 
\addendum{температуры, напряжения и других параметров эксплуатации за допустимые 
границы из-за изменения внешних условий или внешних воздействий.}
%
Выход за допустимые границы может приводить к сбоям и отказам аппаратных 
компонентов, что чревато снижением безопасности.

\addendum{К контролируемым параметрам эксплуатации могут относиться тактовая 
частота, уровень рентгеновского излучения, различные климатические 
факторы~\forref{R.EF.Ranges}.} 

% климатические факторы: ГОСТ 15150-69

\subsection{Требования}\label{EF.Reqs}

\req{ЗВ}{3, 4}\label{R.EF.Ranges}
Должны быть определены допустимые диапазоны температуры внутри корпуса СКЗИ, 
напряжения питания СКЗИ, других \addendum{параметров эксплуатации}, 
влияющих на безопасность СКЗИ. Допустимые диапазоны 
должны быть приведены~\forref{R.GD.Admin,R.GD.Roles} в руководствах.

\req{ЗВ}{4}\label{R.EF.Detect}
Должны быть определены и корректно реализованы механизмы 
обнаружения выхода \addendum{параметров эксплуатации} СКЗИ 
за допустимые границы~\useref{R.EF.Ranges}. 

\req{ЗВ}{4}\label{R.EF.Lock}
При обнаружении выхода за допустимые границы СКЗИ должно переходить
в состояние блокировки~\useref{R.AC.LockState} или полной 
блокировки~\useref{R.AC.CrashState}. 
