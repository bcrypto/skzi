% Guides
\section{Руководства (РД)}\label{GD}

\subsection{Обзор}\label{GD.Intro}

Пакет РД устанавливает требования к руководствам, сопровождающим СКЗИ.

Требования касаются представления в руководствах правил и обязанностей
при работе операторов с СКЗИ. 
%
Требования не касаются дополнительных правил и обязанностей,
а также оформления руководств.
%
Руководства могут оформляться в виде отдельных документов, частей одного 
или нескольких документов.

\subsection{Требования}\label{GD.Reqs}

\req{РД}{1--4}\label{R.GD.Admin} % 1
Должно быть разработано руководство администратора.
Руководство должно описывать:
\begin{itemize}
\item[--]
обязанности администратора по настройке 
среды~\useref{R.ES.Install,R.ES.Objects,R.ES.Session}; 
\item[--]
инструкции по установке~СКЗИ~\useref{R.ES.Install};
\item[--]
доступные администратору сервисы~\useref{R.AC.Policy} 
с указанием допустимых последовательностей их 
вызовов~\useref{R.SV.List};
\item[--]
обязанности администратора по настройке механизмов безопасности~СКЗИ;
\item[--]
связанные с безопасностью предположения 
относительно поведения операторов.
\end{itemize}
Руководство должно описывать обязанности администратора по отслеживанию 
уязвимостей КСК~\useref{R.ES.CVE}, если такое отслеживание предусмотрено.

\req{РД}{1--4}\label{R.GD.Roles} % 2
Для каждой роли~\useref{R.AC.Roles}, отличной от роли <<Администраторы>>, 
должно быть разработано руководство ее операторов.
Руководство должно определять:
\begin{itemize}
\item[--]
доступные оператору сервисы~\useref{R.AC.Policy}
с указанием допустимых последовательностей их 
вызовов~\useref{R.SV.List}; 
\item[--]
обязанности оператора по обеспечению безопасности~СКЗИ.
\end{itemize}

\req{РД}{2--4}\label{R.GD.Misuse} % 3
Руководства~\useref{R.GD.Admin,R.GD.Roles} 
должны описывать типичные ошибки операторов,
которые могут привести к снижению безопасности~СКЗИ.
Руководства должны давать рекомендации операторам по избежанию
ошибок.

