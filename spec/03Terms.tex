\chapter{Термины и определения}\label{TERMS}

В настоящем стандарте применяют  
%
%термины в соответствии с СТБ 34.101.65, а также 
%
следующие термины с соответствующими определениями:

{\bf \thedefctr~аппаратное (аппаратно-программное, 
программно-аппаратное) средство криптографической защиты информации}:
Средство криптографической защиты информации, в состав которого входят 
аппаратные компоненты. 

{\bf \thedefctr~аутентификация}:
Проверка подлинности стороны или данных.

% old: проверка подлинности оператора

% CFRG:
%
% The process of verifying the integrity of data that has been stored, 
% transmitted, or otherwise exposed to possible unauthorized access.
%
% Verifying the identity of a user, process, or device, often as a prerequisite 
% to allowing access to a system’s resources

{\bf \thedefctr~аутентификационные данные}:
Информация, которая используется для аутентификации стороны.

% Information used to verify the claimed identity of a user

\addendum{\bf \thedefctr~бесключевой криптографический алгоритм}:
Криптографический алгоритм, в котором не используются ключи.

% Information used to verify the claimed identity of a user

{\bf \thedefctr~генератор случайных чисел}:
Устройство или программа, которая вырабатывает последовательность 
непредсказуемых элементов.

% aппаратно-программное устройство, 
% вырабатывающее последовательность чисел, 
% каждый следующий элемент которой трудно
% предсказать по всем предыдущим элементам.

{\bf \thedefctr~долговременный объект}:
\addendum{Объект, операции над которым могут выполняться в нескольких сеансах.}

% v32: Объект, который хранится в пределах криптографической границы или 
% передается за ее пределы, операции над которым могут выполняться в нескольких 
% сеансах.

{\bf \thedefctr~зашифрование}:
Преобразование данных, направленное на обеспечение их конфиденциальности,
которое выполняется с использованием секретного или открытого ключа.

{\bf \thedefctr~защита объектов}:
Контроль целостности и подлинности, обеспечение конфиденциальности критических
объектов; контроль целостности и подлинности открытых объектов.

% old: Контроль целостности и обеспечение конфиденциальности
% критических объектов, контроль целостности открытых объектов.

{\bf \thedefctr~идентификация}:
Назначение уникального идентификатора
\addendum{или сравнение предъявляемого идентификатора с назначенными 
идентификаторами}.

% old: Присвоение операторам уникальных идентификаторов.

{\bf \thedefctr~имитовставка}:
Контрольная характеристика данных, которая определяется с использованием
секретного ключа и служит для контроля целостности и подлинности данных.

{\bf \thedefctr~имитозащита}:
Контроль целостности и подлинности данных, который реализуется путем выработки и
проверки имитовставок.

{\bf \thedefctr~клиентская программа}:
\addendum{
Программа, которая использует сервисы одной стороны, выступая от лица другой. 
}

% old: Программа, которая использует сервисы средства криптографической защиты 
% информации [нет функции посредника]. 
%
% old old: Программа, которая вызывает от лица оператора сервисы средства 
% криптографической защиты информации [циклическая ссылка на оператора]

{\bf \thedefctr~конфиденциальность}:
Гарантия того, что данные доступны для использования
только тем сторонам, которым они предназначены.

{\bf \thedefctr~криптографическая граница}: 
Точно определенный разработчиком непрерывный физический периметр в среде
эксплуатации, который определяет контролируемую границу средства
криптографической защиты информации.

% todo: "контролируемую" -- звучит неудачно

{\bf \thedefctr~криптографический алгоритм}:
Алгоритм, который реализует криптографическую операцию, направленную на
обеспечение конфиденциальности, контроль целостности или подлинности данных, в
том числе алгоритм управления \addendum{входными данными (параметрами)} 
других алгоритмов.

\begin{note}
Речь идет об операциях зашифрования и расшифрования, вычисления и проверки
электронной цифровой подписи, \addendum{хэширования}, выработки и проверки 
имитовставки, выработки псевдослучайных данных, формирования совместно 
используемой конфиденциальной информации, разделения секрета, 
построения ключа и др. 
\end{note}

% old: Алгоритм преобразования объектов,
% направленный на обеспечение их конфиденциальности,
% контроля целостности или подлинности,
% в том числе алгоритм управления криптографическими
% ключами для защиты объектов.

% https://csrc.nist.gov/glossary/term/Cryptographic_algorithm:
%  Well-defined procedure or sequence of rules or steps, or a series of 
%  mathematical equations used to describe cryptographic processes such as 
%  encryption/decryption, key generation, authentication, signatures, etc.

\addendum{\bf \thedefctr~криптографический алгоритм с открытым ключом}:
Криптографический алгоритм, который реализует криптографическую
операцию из семейства операций, связанных друг с другом и управляемых двумя 
ключами, один из которых может быть сделан общедоступным, а второй~--- нет.

\addendum{\bf \thedefctr~криптографический алгоритм с секретным ключом}:
Криптографический алгоритм, который реализует криптографическую
операцию из семейства операций, связанных друг с другом и управляемых одним и 
тем же ключом, который не может быть сделан общедоступным.  

{\bf \thedefctr~криптографический ключ}: 
\addendum{Входной} параметр криптографического алгоритма, который управляет 
ходом реализуемой алгоритмом криптографической операции, связан с конкретной 
стороной или группой сторон.

% bign: Параметр, который управляет криптографическими 
% операциями зашифрования и расшифрования, 
% выработки и проверки ЭЦП, генерации псевдослучайных чисел и др.

% old: объект-параметр, используемый вместе с 
% криптографическим алгоритмом для управления операциями зашифрования и 
% расшифрования, вычисления и проверки электронной цифровой подписи, выработки и 
% проверки имитовставки, выработки псевдослучайных данных, выработки совместно 
% используемой конфиденциальной информации.

{\bf \thedefctr~криптографический протокол}:
Интерактивный криптографический алгоритм, который выполняют 
несколько сторон-участников, обмениваясь между собой сообщениями,
содержащими промежуточные результаты вычислений. 

{\bf \thedefctr~криптографический сервис}:
Сервис, реализующий криптографический алгоритм, 
несколько криптографических алгоритмов или отдельные шаги алгоритмов.

{\bf \thedefctr~критический системный компонент; КСК}:
Системный компонент, который влияет на безопасность средства криптографической 
защиты информации.

% old: Находящееся внутри криптографической границы 
% аппаратное и программное обеспечение, которое используется
% для передачи, обработки и хранения объектов
% средства криптографической защиты информации.

{\bf \thedefctr~критический объект}: 
Объект, несанкционированные раскрытие или модификация которого 
снижают безопасность.

{\bf \thedefctr~личный ключ}:
Криптографический ключ, используемый \addendum{в криптографическом алгоритме}
с открытым ключом, который однозначно связан с конкретной стороной и не 
является общедоступным.

{\bf \thedefctr~неявная копия}:
Копия объекта, передаваемая по побочному каналу.

{\bf \thedefctr~объект}: 
Элемент, который содержит или получает информацию
и над которым выполняются операции.

\begin{note}
\addendum{
Примеры объектов: ключи, параметры криптографических алгоритмов, 
аутентификационные данные, сервисы, файлы программ.}
\end{note}

{\bf \thedefctr~оператор}:
Сторона, котор\addendum{ая} использует сервисы средства криптографической 
защиты\addendum{,} или клиентская программа, выступающая от имени стороны.

\addendum{\bf \thedefctr~операционная система}:
Набор программ, который управляет ресурсами аппаратных компонентов внутри 
криптографической границы и обеспечивает унифицированную работу с этими 
ресурсами других программ (приложений).

\begin{note}
\addendum{
Операционная система может быть универсальной, обслуживающей аппаратные 
компоненты и приложения широкого спектра, или специализированной, 
ориентированной на узкий спектр. Крайней формой специализации является
слой системного программного обеспечения, который унифицирует работу
сервисов СКЗИ с конкретными аппаратными компонентами. Допускается отсутствие 
операционной системы и прямая работа сервисов с аппаратными компонентами
(технология <<bare machine>>).
}
\end{note}

\begin{note}
\addendum{
При виртуализации (гостевая) операционная система может работать с
аппаратными компонентами не напрямую, а с помощью другой 
(базовой, хост) операционной системы.
}
\end{note}


{\bf \thedefctr~открытый объект}: 
Объект, несанкционированная модификация которого снижает безопасность, а
раскрытие~--- не снижает.

{\bf \thedefctr~открытый ключ}:
Криптографический ключ, используемый \addendum{в криптографическом алгоритме} 
с открытым ключом, который генерируется вместе с личным ключом и может 
быть сделан общедоступным.

% bign: Ключ, который строится по личному ключу, связан с конкретной стороной,
% может быть сделан общедоступным и используется в настоящем стандарте для
% и для создания токена ключа.
% fix: строится -- некорректно: генерируется пара ключей

\addendum{\bf \thedefctr~пакет (требований)}:
Набор требований безопасности схожего назначения.

{\bf \thedefctr~пароль}:
Секрет, который способен запомнить (обработать) человек и который
поэтому может принимать сравнительно небольшое число значений.

{\bf \thedefctr~побочный канал}:
Нежелательный дополнительный канал передачи информации о 
входных, промежуточных или выходных данных криптографического алгоритма,  
возникающий из-за особенностей его аппаратно\addendum{й и (или)} программной 
реализации. 

{\bf \thedefctr~подлинность}:
Гарантия того, что сторона действительно та, за кого себя выдает; гарантия того,
что сторона действительно является владельцем (создателем, отправителем)
определенных данных.

{\bf \thedefctr~политика управления доступом}:
Правила, определяющие допустимые операции операторов над сервисами, объектами и
критическими системными компонентами.

{\bf \thedefctr~программное средство криптографической защиты информации}:
Средство криптографической защиты информации, выполненное целиком программно, 
без аппаратных компонентов. 

{\bf \thedefctr~разделение секрета}:
Разбиение критического объекта на частичные секреты, 
каждый из которых по отдельности или даже вместе с некоторыми
другими частичными секретами не дает информации об исходном объекте.

{\bf \thedefctr~расшифрование}:
Преобразование, обратное зашифрованию, которое определяется с помощью
секретного или личного ключа.

{\bf \thedefctr~сеанс (оператора)}:
\addendum{Логическая связь между оператором и средством криптографической
защиты информации в течение периода их взаимодействия.}

% old: Период взаимодействия оператора со средством криптографической
% защиты информации.

{\bf \thedefctr~сеансовый объект}:
Объект, который создается, 
используется и уничтожается в течение одного сеанса.

{\bf \thedefctr~секрет аутентификации}:
Пароль, PIN-код и другие аутентификационные данные, которые однозначно связаны с
конкретной стороной и не являются общедоступными.

{\bf \thedefctr~секретный ключ}:
Криптографический ключ, используемый \addendum{в криптографическом алгоритме} 
с секретным ключом, который однозначно связан с конкретной стороной или группой
сторон и не является общедоступным.

% belt: Параметр, который управляет операциями шифрования 
% и имитозащиты и который известен только определенным сторонам.

{\bf \thedefctr~сервис}:
Функция или механизм, реализованные одной стороной для использования другими 
сторонами.

% old: функция, реализованная в средстве криптографической защиты информации 
% и доступная его оператору.
% fix: циклическая ссылка: СКЗИ через сервис, сервис через СКЗИ
%
% https://csrc.nist.gov/glossary/term/service
%  A capability or function provided by an entity.

{\bf \thedefctr~синхропосылка}:
Открытые входные данные криптографического алгоритма,
которые обеспечивают уникальность результатов 
криптографического преобразования на фиксированном ключе.

{\bf \thedefctr~системный компонент}:
Компонент системной среды, который не разрабатыва\addendum{ется} при создании средства
криптографической защиты информации, хотя мо\addendum{жет} быть включен в его состав.

\begin{note}
\addendum{
К системным компонентам относятся
устройства ввода/вывода,
устройства обработки и передачи (процессор, физические интерфейсы),
устройства хранения,
операционная система.
}
\end{note}

\addendum{\bf \thedefctr~системный объект}: 
Объект средства криптографической защиты информации, 
который находится в общем пользовании операторов 
и владельцем которого формально назначается системный оператор.

\begin{note}
\addendum{К системным объектам относятся файлы программ, 
конфигурационные файлы, глобальные неизменяемые объекты.
}
\end{note}

\addendum{\bf \thedefctr~системный оператор}:
Неявный (логический) оператор, который представляет системную среду.

{\bf \thedefctr~системная среда}:
Аппаратные и программные компоненты в пределах криптографической границы
средства криптографической защиты информации.

{\bf \thedefctr~системный сеанс}:
\addendum{Сеанс системного оператора; соответствует непрерывному периоду работы 
средства криптографической защиты информации: от включения (запуска программ) 
до выключения (завершения).}

% old: Непрерывный период работы средства криптографической защиты информации.

{\bf \thedefctr~состояние (средства криптографической защиты информации)}:
Состояние системного сеанса средства.

{\bf \thedefctr~среда эксплуатации}:
Аппаратно\addendum{е и п}рограммное обеспечение, организационные процедуры и 
мероприятия, в рамках которых функционирует средство криптографической защиты 
информации.

{\bf \thedefctr~средство криптографической защиты информации; СКЗИ}:
Набор аппаратных и (или) программных компонентов, который реализует один или 
несколько криптографических сервисов, а также дополнительные функции и 
механизмы, обеспечивающие безопасную работу сервисов.

\begin{note}
В состав средства входят системные компоненты (например, микропроцессор) и
собственные компоненты (встроенное программное обеспечение микропроцессора).
Перечень компонентов определяет разработчик.
\end{note}

% old: Набор аппаратных и программных компонентов, который реализует
% криптографические сервисы, а также возможно функции управления ключами,
% контроля доступа и проверки работоспособности, сопровождающие сервисы.

\addendum{\bf \thedefctr~сторона}:
Активный элемент: лицо, устройство, процесс, сервер, центр, служба.

% https://csrc.nist.gov/glossary/term/entity:
%  An individual, group or an organization participating in an action.
%  An individual (person), organization, device, or process.
%  A human (person/individual/user), organization, device or process.
%  Either a subject (an active element that operates on information or the 
%   system state) or an object (a passive element that contains or receives 
%   information).
% The term user refers to an individual, group, host, domain, trusted 
%  communication channel, network address/port, another netwoik, a remote system 
%  (e.g., operations system), or a process (e.g., service or program) that 
%  accesses the network, or is accessed by it, including any entity that accesses 
%  a network support entity to perform OAM&Prelated tasks.

{\bf \thedefctr~целостность}:
Гарантия того, что данные не изменены при их хранении, передаче или обработке. 

{\bf \thedefctr~шифрование}:
Зашифрование или расшифрование.

{\bf \thedefctr~хэш-значение}:
Контрольная характеристика данных, которая определяется без использования ключа,
служит для контроля целостности данных и для представления данных в (необратимо)
сжатой форме.

{\bf \thedefctr~хэширование}:
Выработка хэш-значений.

{\bf \thedefctr~частичный секрет}:
Критический объект, 
полученный в результате применения метода разделения секрета.

{\bf \thedefctr~электронная цифровая подпись; ЭЦП}:
Контрольная характеристика данных, которая вычисляется с использованием личного
ключа, проверяется с использованием открытого ключа, служит для контроля
целостности и подлинности данных и обеспечивает невозможность отказа от
авторства.

