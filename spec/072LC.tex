% Life Cycle
\section{Поддержка жизненного цикла (ЖЦ)}\label{LC}

\subsection{Обзор}\label{LC.Intro}

Пакет ЖЦ устанавливает требования по управлению жизненным циклом СКЗИ
через систему управления конфигурацией. Система охватывает этапы жизненного 
цикла, связанные с разработкой, поставкой, устранением недостатков.

\subsection{Требования}\label{LC.Reqs}

\req{ЖЦ}{1--4}\label{R.LC.CMSystem}
Должна быть определена и реализована система управления конфигурацией для СКЗИ.
Система должна обеспечивать:
\begin{itemize}
\item[--]
контроль доступа разработчиков к элементам конфигурации;
\item[--]
контроль версий элементов конфигурации;
\item[--]
отслеживание изменений элементов конфигурации;
\item[--]
сборку программ по исходным текстам~\useref{R.DI.Tools}.
\end{itemize}

\req{ЖЦ}{1--4}\label{R.LC.CMList}
В перечень элементов конфигурации должны быть включены:
\begin{itemize}
\item[--]
функциональная спецификация;
\item[--]
программы;
\item[--]
описание программ~\useref{R.DI.ProgSpec};
\item[--]
описание аппаратных компонентов~\useref{R.DI.HardSpec}
(начиная с уровня 3);
\item[--]
исходные тексты программ;
\item[--]
документация по управлению конфигурацией~\useref{R.LC.CMSystem};
\item[--]
документация по поставке~СКЗИ потребителю~\useref{R.LC.Delivery};
\item[--]
документация по устранению недостатков~\useref{R.LC.FlawRemediation}
(начиная с уровня 2);
\item[--]
руководства~\useref{R.GD.Admin,R.GD.Roles}.
\end{itemize}

\req{ЖЦ}{1--4}\label{R.LC.CMVersion}
Каждая версия каждого элемента конфигурации 
должна быть снабжена уникальным идентификатором. 

\req{ЖЦ}{1--4}\label{R.LC.Delivery}
Должна быть определена и реализована система поставки~СКЗИ потребителю.  

\req{ЖЦ}{2--4}\label{R.LC.Authenticode}
Должны быть предусмотрены средства контроля целостности и подлинности 
инсталляционных программ после их доставки потребителю. 

\req{ЖЦ}{2--4}\label{R.LC.FlawRemediation}
Должна быть определена и реализована система устранения недостатков 
в программах и документации~СКЗИ.
Система должна обеспечивать:
\begin{itemize}
\item[--]
регистрацию недостатков;
\item[--]
определение порядка выявления причин недостатков
и исправления недостатков;
\item[--]
отслеживание статуса недостатков 
(подтвержден, исправляется, исправлен и др.);
\item[--]
описание способа устранения недостатков;
\item[--]
порядок извещения потребителей об устранении недостатков.
\end{itemize}

\if0
Обычные этапы жизненного цикла:
\begin{enumerate}[label=\arabic*)]
\item
проектирование и разработка;
\item
выпуск в обращение;
\item
инициализация;
\item
эксплуатация;
\item
вывод из эксплуатации.
\end{enumerate}

В первом этапе участвуют разработчики. На втором этапе организация, 
ответственная за эксплуатацию СКЗИ, выполняет персонализацию СКЗИ и его 
доставку операторам. На третьем этапе администраторы инициализируют СКЗИ~--- 
устанавливают ключи, настраивают политику управления  
доступом и др. На четвертом этапе с СКЗИ работают авторизованные операторы. 
На пятом этапе организация, ответственная за эксплуатацию, 
выводит средство из эксплуатации и выполняет демонтаж устаревшей или сбойной 
аппаратуры. 
\fi
