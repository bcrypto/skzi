% Life Cycle
\section{Поддержка жизненного цикла (ЖЦ)}\label{LC}

\subsection{Обзор}\label{LC.Intro}

Пакет ЖЦ устанавливает требования по управлению жизненным циклом СКЗИ
через систему управления конфигурацией. Система охватывает этапы жизненного 
цикла, связанные с разработкой, поставкой, устранением недостатков.

\subsection{Требования}\label{LC.Reqs}

\req{ЖЦ}{1--4}\label{R.LC.CMSystem}
Должна быть определена и реализована система управления конфигурацией для СКЗИ.
Система должна обеспечивать:
\begin{itemize}
\item[--]
контроль доступа разработчиков к элементам конфигурации;
\item[--]
контроль версий элементов конфигурации;
\item[--]
отслеживание изменений элементов конфигурации;
\item[--]
сборку программ по исходным текстам~\useref{R.DI.Tools}.
\end{itemize}

\req{ЖЦ}{1--4}\label{R.LC.CMList}
В перечень элементов конфигурации должны быть включены:
\begin{itemize}
\item[--]
функциональная спецификация;
\item[--]
программы;
\item[--]
\addendum{описания программ и аппаратных компонентов}~\useref{R.DI.Spec};
\item[--]
исходные тексты программ;
\item[--]
\addendum{конфигурационные файлы средств разработки и сборки 
программ~\useref{R.DI.Tools}};
\item[--]
документация по управлению конфигурацией~\useref{R.LC.CMSystem};
\item[--]
документация по поставке~СКЗИ потребител\addendum{ям}~\useref{R.LC.Delivery};
\item[--]
документация по устранению недостатков~\useref{R.LC.FlawRemediation}
(начиная с уровня 2);
\item[--]
руководства~\useref{R.GD.Admin,R.GD.Roles}.
\end{itemize}

\req{ЖЦ}{1--4}\label{R.LC.CMVersion}
Каждая версия каждого элемента конфигурации 
должна быть снабжена уникальным идентификатором. 

\req{ЖЦ}{1--4}\label{R.LC.Delivery}
Должна быть определен \addendum{порядок} поставки~СКЗИ потребител\addendum{ям}.  

\req{ЖЦ}{2--4}\label{R.LC.Authenticode}
Должны быть предусмотрены средства контроля целостности и подлинности 
инсталляционных программ после их доставки потребител\addendum{ям}. 

\req{ЖЦ}{2--4}\label{R.LC.FlawRemediation}
Должна быть определена и реализована система устранения недостатков в программах
и документации~СКЗИ.
%
Система должна обеспечивать:
\begin{itemize}
\item[--]
регистрацию недостатков;
\item[--]
\addendum{управление} выявлени\addendum{ем} причин недостатков
и исправлени\addendum{ем} недостатков;
\item[--]
отслеживание статуса недостатков 
(подтвержден, исправляется, исправлен и др.);
\item[--]
\addendum{протоколирование} способа устранения недостатков;
\item[--]
\addendum{извещение} потребителей об устранении недостатков.
\end{itemize}

\req{ЖЦ}{1--4}\label{R.LC.SU}
\addendum{
При применении пакета ОП должна быть определена и реализована система 
обновления программ СКЗИ. 
%
Система должна обеспечивать:
}
\begin{itemize}
\item[--]
\addendum{выпуск обновлений};
\item[--]
\addendum{извещение потребителей о выпуске обновлений (с указанием содержания 
обновлений)};  
\item[--]
\addendum{поставку обновлений потребителям}.
\end{itemize}

