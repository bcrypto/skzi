% Trusted Channel
\section{Доверенный канал (ДК)}\label{TC}

\subsection{Обзор}\label{TC.Intro}

Пакет ДК устанавливает требования по защите канала связи между СКЗИ и удаленным 
оператором. 
%
Канал продолжает устройства ввода и вывода СКЗИ в недоверенной среде 
эксплуатации, позволяя оператору проводить обычные сеансы,
как если бы он непосредственно работал со средством.
%
Оператор создает канал и управляет им с помощью клиентской программы. 
Клиентская программа может использовать другие СКЗИ или сама являться СКЗИ.  

В ходе сеанса оператор и СКЗИ обмениваются выходными и выходными 
данными сервисов, а также дополнительно 
текстовыми сообщениями (удаленный терминал), 
данными графического пользовательского интерфейса (удаленный рабочий стол),
кодом (скрипты браузера).
%
При создании канала генерируются ключи, на которых данные обмена защищаются. 
Обычно для генерации ключей используется протокол формирования общего 
ключа. Стороны протокола используют долговременные ключи или 
пароли~\forref{R.TC.Crypto}.

Среди входных и выходных данных сервисов могут быть объекты СКЗИ 
перечня~\useref{R.AC.Objects}. Передача объектов между оператором и средством 
считается экспортом~/ импортом, и поэтому объекты обязательно защищаются с помощью 
криптографических методов~\useref{R.DP.CryptoE,R.DP.CryptoI}. Защита объектов 
может быть совмещена с защитой других данных обмена, может выполняться 
независимо или дополнительно~\forref{R.TC.Crypto}.

Оператор проходит аутентификацию перед СКЗИ и, кроме этого, проводит встречную 
аутентификацию. Встречная аутентификация защищает от атак <<противник 
посередине>>. Она может быть частью протокола формирования общего 
ключа~\forref{R.TC.Auth2}.

Существует угроза определения текущих данных обмена в будущем, после раскрытия 
долговременных ключей или паролей. Для противодействия угрозе обеспечивается 
защита от <<чтения назад>>. Обычно для этого в протокол формирования общего ключа
вводят дополнительные сеансовые ключи~\forref{R.TC.FSWeak,R.TC.FS}. 

Обычный способ реализации доверенного канала состоит в применении протокола 
TLS, определенного в СТБ 34.101.65. При этом способе криптографическая
защита канала регулируется криптонаборами TLS, которые представляют
собой взаимосвязанные криптографические алгоритмы разного назначения.

\subsection{Требования}\label{TC.Reqs}

\req{ДК}{1--4}\label{R.TC.GD}
В руководствах должна быть~\forref{R.GD.Admin,R.GD.Roles}
представлена информация о настройке клиентской программы, которая используется 
для связи между удаленным оператором и СКЗИ.

\req{ДК}{1--4}\label{R.TC.Auth}
Аутентификация удаленного оператора должна~\forref{R.IA.Auth} 
выполняться средствами СКЗИ.

\req{ДК}{1--4}\label{R.TC.Auth2}
Должна проводиться встречная аутентификация СКЗИ перед удаленным оператором.
Вероятность пройти аутентификацию, не зная секрета аутентификации, 
не должна превышать~$2^{-64}$.

\req{ДК}{1--4}\label{R.TC.Logout}
У удаленного оператора должна быть возможность завершить сеанс в любой момент 
времени. Сеанс должен автоматически завершаться при отсутствии активности 
оператора в течение определенного времени.

\req{ДК}{1--4}\label{R.TC.Crypto}
Должны быть обеспечены конфиденциальность, контроль целостности и подлинности 
данных обмена между удаленным оператором и сервисами СКЗИ.
%
Для защиты должны использоваться криптографические методы.
%
Ключи защиты должны обновляться в каждом новом сеансе оператора.

\begin{note}
Сеанс оператора можно сохранить (кэшировать), а затем возобновить.
При возобновлении сеанса ключи защиты обновлять не обязательно.
\end{note}

\req{ДК}{3, 4}\label{R.TC.CryptoBY}
Для защиты должны использоваться криптографические 
методы~\useref{R.DP.CryptoE,R.DP.CryptoI}. 

\begin{note}
В~\useref{R.TC.Crypto} не обязательно использовать методы,
определенные в~\useref{R.DP.CryptoE,R.DP.CryptoI}. Это значит, что для защиты 
данных обмена (исключая объекты СКЗИ) могут использоваться внешние 
криптографические средства, например, протокол TLS с криптонаборами, 
не установленными в ТНПА и не реализованными в СКЗИ. 
%
Внешние криптографические средства обязательно входят в состав 
КСК~\useref{R.ST.CSCList} и подпадают под контроль.
\end{note}

\req{ДК}{2--4}\label{R.TC.Keys}
В формировании ключей защиты~\useref{R.TC.Crypto} должен участвовать СКЗИ.

\begin{note}
Недопустима ситуация, когда СКЗИ не влияет на формирование ключей защиты.
Например, когда их генерирует удаленный оператор и пересылает 
в зашифровыванном на открытом ключе СКЗИ виде.
\end{note}

\req{ДК}{1--4}\label{R.TC.Pwd}
Если ключи защиты~\useref{R.TC.Crypto} формируются на основании пароля
удаленного оператора, то этот пароль должно быть вычислительно
трудно определить по данным обмена между оператором и СКЗИ.

\req{ДК}{3, 4}\label{R.TC.FSWeak}
Ключи защиты~\useref{R.TC.Crypto} должны формироваться так, чтобы их трудно
было определить даже после раскрытия долговременного ключа или пароля удаленного 
оператора.

\req{ДК}{4}\label{R.TC.FS}
Ключи защиты~\useref{R.TC.Crypto} должны формироваться так, 
чтобы их трудно было определить даже после раскрытия долговременного ключа 
СКЗИ.

\req{ДК}{3, 4}\label{R.TC.AU}
Открытие и закрытие доверенного канала должны~\forref{R.AU.Events} 
регистрироваться в журнале аудита. Запись аудита должна идентифицировать 
клиентскую программу.



