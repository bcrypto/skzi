\providecommand{\CryptoKernel}{<<{Криптоядро}>>\xspace}
\begin{appendix}{Б}{справочное}{Программное средство \CryptoDisk 
(примерная спецификация)}
\label{EXAMPLE} 

\hiddensection{Описание}

\paragraph*{Назначение.} 
Программное средство~\CryptoDisk предназначено для шифрования 
файлов на магнитных носителях <<на лету>>. 
%
С помощью программного средства пользователи могут создавать частные каталоги,
при записи в которые файлы будут зашифровываться, 
а при чтении~--- расшифровываться.
%
Сам каталог без обработки~программным средством представляет 
собой обычный файл операционной системы. 
%
Пользователи, которым запрещено чтение каталога,
не могут раскрыть его содержание даже при доступе к данному файлу.

\paragraph*{Класс.} 
\CryptoDisk является средством класса~$2$.

\paragraph*{Основные функциональные возможности.} 
\CryptoDisk реализует:
\begin{itemize}
\item[--]
управление частными каталогами нескольких пользователей;
\item[--]
шифрование файлов каталога <<на лету>>;
\item[--]
хранение ключей на USB-токенах;
\item[--]
возможность записи файлов в каталог 
любым авторизованным пользователем;
\item[--]
установку разрешений на доступ к отдельным файлам 
каталога совместно администратору и выбранному доверенному пользователю 
(на случай утери или блокировки токена владельца).
\end{itemize}

\paragraph*{Криптографическая граница.}
\CryptoDisk выполняется на персональном компьютере. 
Криптографической границей является периметр системного 
блока компьютера.

\paragraph*{Критические системные компоненты.}
В пределах криптографической границы находятся следующие критические 
системные компоненты:

\begin{definition}{CSC.Board}{Вычислительная платформа}
Процессор, материнская плата, устройства хранения, 
порты ввода/вывода и другие аппаратные 
компоненты персонального компьютера, 
необходимые для выполнения программ и хранения данных~\CryptoDisk.
%
Должны быть доступны два порта для USB-токенов.
%
Должен присутствовать высокоточный таймер.
\end{definition}

\begin{definition}{CSC.OS}{Операционная система}
Компоненты операционной системы, 
отвечающие за безопасное выполнение программ~\CryptoDisk. 
Должна использоваться операционная система линейки~Windows.
\end{definition}

\hiddensection{Криптографическая поддержка}\label{EXAMPLE.Crypto}

\paragraph*{Криптографические алгоритмы.} 
В~\CryptoDisk
реализованы следующие криптографические алгоритмы:

\begin{definition}{Alg.DataWrap}{Алгоритмы одновременного шифрования и имитозащиты}
Алгоритмы одновременного шифрования и имитозащиты, определенные в~ТНПА-1. 
Реализованы алгоритм установки защиты 
(зашифрование и вычисление имитовставки) 
и алгоритм снятия защиты
(проверка имитовставки и расшифрование).
Алгоритмы предназначены для защиты файлов в частных каталогах.
\end{definition}

\begin{definition}{Alg.Transport}{Алгоритмы транспорта ключей} 
Алгоритмы транспорта ключей, определенные в ТНПА-2. 
Реализованы алгоритм установки защиты транспортируемого
ключа и алгоритм снятия защиты. 
При установке защиты используется открытый ключ оператора, 
которому транспортируется ключ,
а при снятии защиты используется личный ключ того же оператора.
Алгоритмы предназначены для транспорта ключей~\ref{Alg.DataWrap}.
\end{definition}

\begin{definition}{Alg.Sign}{Алгоритмы ЭЦП} 
Алгоритмы электронной цифровой подписи, определенные в ТНПА-3. 
Реализованы алгоритмы выработки и проверки ЭЦП.
Используется администратором для заверения открытых 
ключей~\ref{Alg.Transport} и пользователями для проверки заверения.
\end{definition}

\begin{definition}{Alg.PRNG}{Алгоритм генерации псевдослучайных чисел} 
Алгоритм генерации псевдослучайных чисел с секретным параметром, 
определенный в ТНПА-4. 
Алгоритм предначен для генерации ключей~\ref{Alg.DataWrap} и~\ref{Alg.Transport}, 
других случайных параметров криптографических алгоритмов.
\end{definition}

\begin{definition}{Alg.Hash}{Алгоритм хэширования} 
Алгоритм хэширования, определенный в ТНПА-5. 
Алгоритм предначен для вычисления контрольных характеристик 
системных объектов. Является композиционным 
элементом~\ref{Alg.Transport} и~\ref{Alg.Sign}.
Используется в генераторе случайных чисел.
\end{definition}

\paragraph*{Генерация ключей и параметров.} 
Алгоритмы~\ref{Alg.DataWrap}, \ref{Alg.PRNG} и \ref{Alg.Hash}
не имеют долговременных параметров. 
%
Долговременные параметры~\ref{Alg.Transport} и~\ref{Alg.Sign} совпадают.
К ним относятся параметры эллиптической кривой и базовая точка на ней. 
Эти параметры генерируются по алгоритму, определенному в ТНПА-3, 
фиксируются в программах~\CryptoDisk и используются всеми операторами.

Ключ~\ref{Alg.PRNG} вырабатывается с помощью генератора случайных чисел~(Б.7). 
После создания ключ записывается на токен вместе с нулевым счетчиком. 
Дальнейшая работа с~\ref{Alg.PRNG} регулируется следующими правилами:
\begin{enumerate} 
\item
При обращении к~\ref{Alg.PRNG} ключ и счетчик читаются с токена. 
По ним вырабатываются псевдослучайные данные требуемого объема. 
\item
При выработке данных счетчик изменяется.
Новое значение счетчика записывается на токен. 
\item
Если запись прошла успешно,
то сформированные псевдослучайные данные используются по назначению. 
В противном случае данные уничтожаются 
и обращение к~\ref{Alg.PRNG} завершается ошибкой.
\end{enumerate} 

По данным правилам вырабатываются 
секретные ключи и синхропосылки~\ref{Alg.DataWrap}, 
личные ключи и секретные параметры~\ref{Alg.Transport} и~\ref{Alg.Sign},
частичные секреты и др.

\clearpage

Открытые ключи~\ref{Alg.Transport} и~\ref{Alg.Sign} 
вырабатываются по личному ключу в соответствии с алгоритмами, 
определенными в~ТНПА-2, ТНПА-3.

\paragraph*{Контроль времени выполнения.} 
Алгоритмы~\ref{Alg.DataWrap} и \ref{Alg.PRNG} относятся к классу симметричных.
Время их выполнения не зависит от значений ключей.

Алгоритм~\ref{Alg.Hash} является бесключевым.
При хэшировании объектов, которые могут оказаться критическими, 
время хэширования не зависит от значения объекта
и определяется только размером объекта.

В алгоритмах~\ref{Alg.Transport}, \ref{Alg.Sign}
определяются кратные 
$uP=P+P+\ldots+P$, где $P$~--- точка эллиптической кривой,
которая суммируется $u$ раз.
При этом кратность~$u$ может являться личным ключом 
или одноразовым секретным параметром.
%
Для определения кратных точек используется 
алгоритм <<лестница Монтгомери>> (Montgomery ladder),
время работы которого не зависит от кратности.

\hiddensection{Реализация сервисов}

\paragraph*{Сервисы.} 
\CryptoDisk реализует следующие сервисы:

\begin{definition}{S.GetVersion}{Номер версии}
Сервис отображает номер версии программ в формате 
<<старший номер, младший номер, номер сборки>>. 
\end{definition}

\begin{definition}{S.SelfTest}{Самотестирование}
Сервис самотестирования.
\end{definition}

\begin{definition}{S.InitToken}{Инициализация токена}
На вход сервиса подается идентификатор оператора,
которому токен передается во владение.
%
Сервис настраивает файловую систему токена,
записывает на токен идентификатор и служебные данные.
%
Если оператор не является администратором, 
то дополнительно используется уже инициализированный токен администратора,
%
с которого переписывается ключ~\ref{R.SignKeyPublic}. 
\end{definition}

\begin{definition}{S.GenUserKeys}{Генерация ключей пользователя}
Сервис запрашивает PIN-код доступа и задает его для токена.
После этого для доступа к объектам токена требуется предъявить PIN-код.
%
Сервис генерирует ключи~\ref{R.KeyPRNG}, 
формирует нулевой счетчик~\ref{R.CounterPRNG}
и записывает эти объекты на токен. 
Затем генерируются и записываются на токен ключи~\ref{R.TransportKeyPrivate}
и~\ref{R.TransportKeyPublic}.
%
Ключ~\ref{R.TransportKeyPublic} дополнительно записывается 
в каталог администратора вместе с идентификатором владельца токена.
\end{definition}

\begin{definition}{S.GenAdminKeys}{Генерация ключей администратора}
Сервис выполняет те же действия, что и~\ref{S.GenUserKeys}.
Дополнительно генерируются и записываются на токен ключи~\ref{R.SignKeyPrivate}
и~\ref{R.SignKeyPublic}. Дополнительно вызывается 
сервис~\ref{S.SignTransportKey}, на вход которого подаются
ключ~\ref{R.TransportKeyPublic} и идентификатор администратора.
\end{definition}

\clearpage
\begin{definition}{S.AuthToken}{Аутентификация для доступа к токену}
На вход сервиса подается PIN-код оператора.
PIN-код передается управляющей программе токена. 
После подтверждения PIN-кода оператору разрешается доступ к объектам токена.
\end{definition}

\begin{definition}{S.SignTransportKey}{Подпись ключа транспорта}
На вход сервиса подается запись,
содержащая ключ~\ref{R.TransportKeyPublic}
и идентификатор его владельца.
%
Используется токен администратора.
%
Сервис читает с токена ключ~\ref{R.SignKeyPrivate},
вычисляет на нем ЭЦП записи и добавляет запись вместе с ЭЦП 
в общедоступный справочник открытых ключей.
\end{definition}

\begin{definition}{S.WriteFile}{Запись файла}
На вход сервиса подается файл, который пользователю~$U$
требуется записать в каталог пользователя~$H$.
%
Используется токен~$U$.
Из справочника открытых ключей читается запись,
содержащая идентификатор~$H$ и его ключ~\ref{R.TransportKeyPublic}.
ЭЦП этой записи проверяется на ключе~\ref{R.SignKeyPublic},
который читается с токена~$U$. 
Если проверка прошла успешно, 
то генерируется ключ~\ref{R.DataKey} и на нем устанавливается 
защита входного файла.
%
Ключ~\ref{R.DataKey} защищается на ключе~\ref{R.TransportKeyPublic}
пользователя~$U$ и сохраняется вместе с зашифрованным файлом в каталоге. 
\end{definition}

\begin{definition}{S.ReadFile}{Чтение файла}
На вход сервиса подается зашифрованный файл из частного каталога
и зашифрованный ключ~\ref{R.DataKey} его защиты. 
%
Используется токен владельца каталога.
%
С токена читается ключ~\ref{R.TransportKeyPrivate}
и на нем снимается защита с ключа~\ref{R.DataKey}.
Затем на ключе~\ref{R.DataKey} снимается защита с зашифрованного файла.
\end{definition}

\begin{definition}{S.ShareFile}{Открытие доступа к файлу}
На вход сервиса подается зашифрованный ключ~\ref{R.DataKey},
который хранится вместе с зашифрованным файлом в частном каталоге,
и идентификатор доверенного пользователя, которому совместно с администратором 
предоставляется доступ к файлу.
%
Используется токен владельца каталога.
%
С токена читается ключ~\ref{R.TransportKeyPrivate}
и на нем снимается защита с ключа~\ref{R.DataKey}.
Затем ключ~\ref{R.DataKey} разбивается на два частичных 
секрета~\ref{R.DataKeyPartial}.
%
Каждый из секретов защищается на ключах~\ref{R.TransportKeyPublic} 
доверенного пользователя и администратора. 
%
Ключ~\ref{R.TransportKeyPublic} доверенного пользователя 
читается из общедоступного справочника. Перед использованием 
проверяется его ЭЦП.
%
Зашифрованные частичные секреты сохраняются вместе с файлом в каталоге.
\end{definition}

\begin{definition}{S.RecoverFile}{Восстановление файла}
На вход сервиса подается зашифрованный файл из частного каталога
и зашифрованные частичные секреты~\ref{R.DataKeyPartial}
пользователя и администратора, 
которым совместно открыт доступ к файлу в частном каталоге.
Используются токены доверенного пользователя и администратора.
%
С токенов читаются ключи~\ref{R.TransportKeyPrivate}.
На них снимается защита с~\ref{R.DataKeyPartial}. 
По найденным частичным секретам собирается ключ~\ref{R.DataKey}, 
на котором снимается защита с целевого файла.
\end{definition}

Все сервисы возвращают код ошибки или признак успешного завершения.

Сервисы \ref{S.InitToken}, \ref{S.GenAdminKeys}, \ref{S.GenUserKeys}, 
\ref{S.SignTransportKey} обслуживают управление ключами  
и выполняются в следующей последовательности:

\begin{enumerate}
\item
Администратор генерирует свои
ключи, вызывая~\ref{S.InitToken} и~\ref{S.GenAdminKeys}.

\item
По запросу пользователя администратор инициализирует для него токен,
вызывая~\ref{S.InitToken}. Токен передается пользователю.

\item
Пользователь генерирует ключи, вызывая~\ref{S.GenUserKeys}.
Открытый ключ~\ref{R.TransportKeyPublic} 
помещается в каталог администратора.

\item
Администратор с определенной периодичностью просматривает свой каталог
и подписывает новые записи (идентификатор пользователя, 
открытый ключ пользователя), 
вызывая~\ref{S.SignTransportKey}.
При необходимости перед вызовом сервиса администратор 
проверяет соответствие между идентификатором
и открытым ключом, связываясь с пользователем.
\end{enumerate}

\paragraph*{Защита от создания неявных копий критических объектов.}
Неявные копии критических объектов 
могут создаваться в следующих областях памяти:
\begin{itemize}
\item[--]
файл подкачки (\texttt{pagefile.sys}); 
\item[--]
файл спящего режима (\texttt{hiberfile.sys});
\item[--]
отладочные дампы памяти, сохраняемые операционной системой 
при сбоях в программах.
\end{itemize}
В~\CryptoDisk используется специальные функции выделения
оперативной памяти, которые блокируют попадание критических сеансовых объектов
в файл подкачки.
Дополнительно при настройке среды блокируются спящий режим и 
средства создания отладочных дампов памяти.

\hiddensection{Управление доступом}

\paragraph*{Роли.} 
\CryptoDisk поддерживает роли~\anchor{Role.Admins} (<<Администраторы>>)
и \anchor{Role.Users} (<<Пользователи>>).
Группа~\ref{Role.Admins} включает только одного участника.

\paragraph*{Объекты.} 
Обрабатываются следующие объекты:

\begin{definition}{R.File}{Файл}
Файл частного каталога.
Долговременный критический объект.
Принадлежит владельцу каталога.
\end{definition}

\begin{definition}{R.KeyPRNG}{Секретный ключ~ГПСЧ}
Секретный ключ алгоритма~\ref{Alg.PRNG}.
Вырабатывается с помощью генератора случайных чисел.
Хранится на токене. Принадлежит владельцу токена.
\end{definition}

\begin{definition}{R.CounterPRNG}{Счетчик~ГПСЧ}
Счетчик алгоритма~\ref{Alg.PRNG}. Открытый объект.
Первоначально устанавливается в~$0$, при обращениях к~\ref{Alg.PRNG}
последовательно увеличивается. Хранится на токене. 
Принадлежит владельцу токена.
\end{definition}

%\begin{definition}{R.AuthData}{Контрольные аутентификационные данные}
%Контрольная характеристика пароля доступа к токену.
%Долговременный критический объект.
%Результат $1024$-кратного применения~\ref{Alg.Hash} 
%к паролю пользователя, дополненному случайным числом.
%Хранится на токене вместе с использованным случайным числом.
%Принадлежит владельцу токена.
%\end{definition}

\begin{definition}{R.DataKey}{Ключ защиты данных}
Секретный ключ алгоритмов~\ref{Alg.DataWrap} для защиты~\ref{R.File}.
Генерируется с помощью~\ref{Alg.PRNG}.
Хранится вместе с защищенным файлом. 
Принадлежит владельцу~\ref{R.File}.
\end{definition}

\begin{definition}{R.DataKeyPartial}{Частичный ключ защиты данных}
Секретный частичный ключ алгоритмов~\ref{Alg.DataWrap}.
Владельцем является администратор или один из пользователей.
Частичный ключ администратора генерируется с помощью~\ref{Alg.PRNG}.
Частичный ключ пользователя определяется как сумма~\ref{R.DataKey} 
и частичного ключа администратора. 
Хранится вместе с защищенным файлом. 
\end{definition}

\begin{definition}{R.SignKeyPrivate}{Личный ключ ЭЦП}
Личный ключ алгоритма выработки ЭЦП~\ref{Alg.Sign}.
Генерируются с помощью~\ref{Alg.PRNG}.
Хранится на токене администратора. Принадлежит администратору.
\end{definition}

\begin{definition}{R.SignKeyPublic}{Открытый ключ ЭЦП}
Открытый ключ алгоритма проверки ЭЦП~\ref{Alg.Sign}.
Генерируется по личному ключу~\ref{R.SignKeyPrivate}.
Хранится на токене администратора и на токенах пользователей. 
Принадлежит администратору.
\end{definition}

\begin{definition}{R.TransportKeyPrivate}{Личный ключ транспорта}
Личный ключ алгоритма~\ref{Alg.Transport}.
Генерируется с помощью~\ref{Alg.PRNG}.
Хранится на токене. Принадлежит владельцу токена.
\end{definition}

\begin{definition}{R.TransportKeyPublic}{Открытый ключ транспорта}
Открытый ключ алгоритма~\ref{Alg.Transport}.
Генерируется по личному ключу~\ref{R.TransportKeyPrivate}.
Хранится на токене и в общедоступном справочнике открытых ключей. 
Принадлежит владельцу токена.
\end{definition}

\begin{definition}{R.System}{Системные объекты}
Программы, файлы настроек, общедоступный справочник открытых ключей,
данные для тестирования криптографических алгоритмов, флаг блокировки.
Являются открытыми объектами.
Хранятся в пределах криптографической границы.
\end{definition}

\paragraph*{Политика управления доступом.} 
Политика управления доступом определена в таблице~\ref{Table.CryptoDisk.DAC}.
В таблице операции над объектами обозначаются следующим образом:
\texttt{X}~--- выполнение,
\texttt{C}~--- создание,
\texttt{W}~--- запись,
\texttt{R}~--- чтение.
%
Владельцами объектов, над которыми выполняются операции,
в основном являются операторы, вызывающие сервисы. 
%
Исключение составляют системные объекты, а также объекты, 
снабженные надстрочными символами:
$A$~--- объект администратора, $U$~--- объект другого пользователя, 
$H$~--- объект владельца каталога.

\begin{table}[p]
\caption{Политика управления доступом \CryptoDisk}
\label{Table.CryptoDisk.DAC}
\begin{tabular}{|p{5cm}|p{8.7cm}|c|}
\hline
Операторы и сервисы & Объекты & Операции\\
\hline
\hline
\ref{Role.Admins},        & \ref{S.GetVersion} & \texttt{X}\\
\ref{Role.Users}          & \ref{S.SelfTest} & \texttt{X}\\
                          & \ref{S.GenUserKeys} & \texttt{X}\\
                          & \ref{S.AuthToken} & \texttt{X}\\
                          & \ref{S.ReadFile} & \texttt{X}\\
                          & \ref{S.RecoverFile} & \texttt{X}\\
\hline
\ref{Role.Admins}         & \ref{S.InitToken} & \texttt{X}\\
                          & \ref{S.GenAdminKeys} & \texttt{X}\\
                          & \ref{S.SignTransportKey} & \texttt{X}\\
\hline
\ref{Role.Users}          & \ref{S.WriteFile} & \texttt{X}\\
                          & \ref{S.ShareFile} & \texttt{X}\\
\hline
\hline
\ref{S.GetVersion}& \ref{R.System} (данные о версии) & \texttt{R}\\
\hline
\ref{S.SelfTest}& \ref{R.System} (тестовые данные) & \texttt{R}\\
\hline
\ref{S.InitToken} & \ref{R.SignKeyPublic} & \texttt{W}\\
\hline
\ref{S.GenUserKeys}  & \ref{R.KeyPRNG}& \texttt{CW}\\
                     & \ref{R.CounterPRNG}& \texttt{CW}\\
                     & \ref{R.TransportKeyPrivate}& \texttt{CW}\\
                     & \ref{R.TransportKeyPublic}& \texttt{CW}\\
\hline
\ref{S.GenAdminKeys} & \ref{S.GenUserKeys}& \texttt{X}\\
                     & \ref{R.SignKeyPrivate}& \texttt{CW}\\
                     & \ref{R.SignKeyPublic}& \texttt{CW}\\
                     & \ref{S.SignTransportKey}& \texttt{X}\\
\hline
\ref{S.AuthToken}    & --& \\
\hline
\ref{S.SignTransportKey} & \ref{R.TransportKeyPublic}\up{U} & \texttt{RW}\\
                          & \ref{R.SignKeyPrivate} & \texttt{R}\\
\hline
\ref{S.WriteFile} & \ref{R.TransportKeyPublic}\up{H} & \texttt{R}\\
                     & \ref{R.SignKeyPublic}\up{A} & \texttt{R}\\
                     & \ref{R.DataKey}\up{H} & \texttt{CW}\\
                     & \ref{R.File}\up{H} & \texttt{W}\\
\hline
\ref{S.ReadFile}  & \ref{R.File}& \texttt{R}\\
                     & \ref{R.DataKey}& \texttt{R}\\
                     & \ref{R.TransportKeyPrivate}& \texttt{R}\\
\hline
\ref{S.ShareFile} & \ref{R.DataKey}& \texttt{R}\\
                     & \ref{R.TransportKeyPrivate} & \texttt{R}\\
                     & \ref{R.DataKeyPartial}\up{UA} & \texttt{CW}\\
                     & \ref{R.TransportKeyPublic}\up{UA} & \texttt{R}\\
                     & \ref{R.SignKeyPublic}\up{A} & \texttt{R}\\
\hline
\ref{S.RecoverFile} & \ref{R.DataKeyPartial} & \texttt{R}\\
                     & \ref{R.TransportKeyPrivate} & \texttt{R}\\
                     & \ref{R.DataKey}\up{H} & \texttt{C}\\
                     & \ref{R.File}\up{H} & \texttt{R}\\
\hline
\end{tabular}
\end{table}

\paragraph*{Состояния.} 
Главная программа~\CryptoDisk реализует службу операционной системы Windows.
Выполнение службы начинается при запуске системы и заканчивается при ее завершении.
Системный сеанс~\CryptoDisk соответствует периоду выполнения службы.

Используются следующие состояния системного сеанса 
и правила перехода между состояниями:

\clearpage
\begin{definition}{State.Start}{Загрузка}
Состояние загрузки. 
При загрузке проверяется флаг блокировки, 
который хранится в защищенном каталоге
(изменение флага разрешено только системе и администраторам).
Если флаг установлен, то выполняется переход в состояние~\ref{State.Lock},
иначе~--- самотестирование.
Если во время тестирования произошла ошибка, 
то снова выполняется переход в~\ref{State.Lock}.
При успешном тестировании разрешается выполнить сервис~\ref{S.AuthToken}.
При успешной аутентификации выполняется переход 
в состояние~\ref{State.Admin} или~\ref{State.User}.
\end{definition}

\begin{definition}{State.Admin}{Сеанс администратора}
Состояние, которое соответствует сеансу администратора.
Разрешено выполнять сервисы администратора.
При ошибках и сбоях выполняется переход в состояние~\ref{State.Lock}.
\end{definition}

\begin{definition}{State.User}{Сеанс пользователя}
Состояние, которое соответствует сеансу пользователя.
Разрешено выполнять сервисы пользователя.
При ошибках и сбоях выполняется переход в состояние~\ref{State.Lock}.
\end{definition}

\begin{definition}{State.Lock}{Блокировка}
Состояние блокировки, из которого \TOE может быть выведен администратором.
При блокировке завершаются все сервисы, закрываются файлы,
устанавливается флаг блокировки.
Разрешено выполнять сервис аутентификации~\ref{S.AuthToken} на роль~\ref{Role.Admins}.
При успешной аутентификации выполняется переход в состояние~\ref{State.Admin}. 
\end{definition}

\hiddensection{Защита объектов}

\paragraph*{Криптографические методы защиты.}
Для обеспечения конфиденциальности и контроля целостности~\ref{R.File} 
применяется алгоритм~\ref{Alg.DataWrap}.
Используются ключи~\ref{R.DataKey}, свои для каждого~\ref{R.File}.
Синхропосылки алгоритма выбираются случайно с помощью алгоритма~\ref{Alg.PRNG}
и сохраняются вместе с защищенным~\ref{R.File}.

Для обеспечения конфиденциальности и контроля целостности 
ключей~\ref{R.DataKey} и частичных секретов~\ref{R.DataKeyPartial}
применяются алгоритмы~\ref{Alg.Transport}.
При установке защиты используются ключи~\ref{R.TransportKeyPublic},
при снятии защиты~--- ключи~\ref{R.TransportKeyPrivate}.

Для контроля целостности ключей~\ref{R.TransportKeyPublic},
размещенных в общедоступном справочнике,
применяются алгоритмы ЭЦП~\ref{Alg.Sign}.
При выработке ЭЦП используется личный ключ администратора~\ref{R.SignKeyPrivate}. 
При проверке ЭЦП используется открытый ключ~\ref{R.SignKeyPublic},
размещенный на токенах пользователей.

\paragraph*{Аппаратные методы защиты.}
Аппаратные методы защиты реализуются применением USB-токенов,
удовлетворяющих ТНПА-6.
В защищенной памяти токена размещаются следующие ключи его владельца:
\ref{R.CounterPRNG}, \ref{R.KeyPRNG}, \ref{R.TransportKeyPrivate}, 
\ref{R.TransportKeyPublic}.
Дополнительно на токене администратора размещаются его ключи 
\ref{R.SignKeyPrivate}, \ref{R.SignKeyPublic}, 
а на токенах пользователей~--- ключ \ref{R.SignKeyPublic} администратора.

Доступ к ключам регулируется управляющей программой токена.
Прежде чем получить доступ, требуется пройти аутентификацию
с помощью сервиса~\ref{S.AuthToken}.

\paragraph*{Методы разделения секрета.}
Ключ~\ref{R.DataKey} разделяется на два частичных секрета~\ref{R.DataKeyPartial},
которые передаются администратору и доверенному пользователю.
Ключ и частичные секреты являются двоичными строками одинаковой длины.
Частичный секрет администратора выбирается случайно с помощью алгоритма~\ref{Alg.PRNG}.
Второй частичный секрет определяется как сумма (поразрядная по модулю~$2$)
первого с~\ref{R.DataKey}.

\paragraph*{Алгоритмические методы контроля целостности.}
Для контроля~\ref{R.System} используются контрольные хэш-значения,
которые фиксируются в программах~\CryptoDisk.
Хэш-значения вычисляются по алгоритму~\ref{Alg.Hash}.

\paragraph*{Очистка критических сеансовых объектов.}
Критические сеансовые объекты размещаются в оперативной памяти компьютера.
Очистка состоит в обнулении соответствующих областей памяти.
Очистка выполняется после использования объекта или 
при возникновении исключительной ситуации в ходе выполнения программ.

\hiddensection{Самотестирование}

\paragraph*{Работоспособность критических системных компонентов.}
При инсталляции~\CryptoDisk проверяется, что
\begin{itemize}
\item[--]
используется одна из следующих операционных систем:
Windows Vista, Windows 7, Windows Server 2008, 
Windows Server 2008 R2;
\item[--]
процессор имеет $64$-разрядный регистр-таймер \texttt{TSC} 
(time stamp counter),
содержимое которого увеличивается на каждом такте работы;
\item[--]
частота процессора не ниже $600$~MГц.
\end{itemize}

При каждом запуске~\CryptoDisk 
в состоянии~\ref{State.Start} проверяется, что
\begin{itemize}
\item[--]
спящий режим (hibernation) отключен;
%
\item[--]
средства создания отладочных дампов памяти отключены.
\end{itemize}

При чтении данных с токенов проверяется работоспособность 
USB-портов.

\paragraph*{Самотестирование.}
В состоянии~\ref{State.Start} и по запросу оператора выполняются следующие проверки:
\begin{itemize}
\item[--]
тесты известного ответа для~\ref{Alg.DataWrap}, \ref{Alg.Transport}, 
\ref{Alg.Sign}, \ref{Alg.PRNG}, \ref{Alg.Hash};
\item[--]
тесты прямого и обратного преобразований для~\ref{Alg.DataWrap}, \ref{Alg.Transport}, 
\ref{Alg.Sign};
\item[--]
тесты на соответствие между~\ref{R.TransportKeyPrivate} и \ref{R.TransportKeyPublic}, 
между~\ref{R.SignKeyPrivate} и \ref{R.SignKeyPublic}. 
\end{itemize}

Перед генерацией личных и секретных ключей проводится тестирование
выходных последовательностей генератора случайных чисел.

\paragraph*{Ошибки самотестирования.}
При ошибках самотестирования выполняется переход в состояние~\ref{State.Lock}.

\hiddensection{Генерация случайных чисел}\label{EXAMPLE.RNG}

\paragraph*{Генератор случайных чисел.}
Используется генератор случайных чисел с двумя источниками случайности.
Первым источником являются временные интервалы между нажатиями 
оператором клавиш на клавиатуре (клавиатурный источник).
%
Вторым источником является тепловый шум в аналоговых цепях токена (физический источник).

При нажатии клавиш фиксируются значения регистра~\texttt{TSC}.
Разность между значениями регистра сохраняется,
если друг за другом нажаты две различные клавиши и интервал между нажатиями более $50$~мс. 
Всего сохраняется $128$ разностей.
Из них составляется $1024$-битовая строка.

Микроконтроллер токена оцифровывает тепловой шум и формирует $16$-битовое 
случайное слово. По $16$ обращениям к микроконтроллеру формируется $256$-битовая строка.

Строки, полученные от двух источников случайности, объединяются и хэшируются.
Полученное хэш-значение считается выходом генератора.

\paragraph*{Оценка энтропии.}
Были проведены вычислительные эксперименты,
направленные на оценку энтропии выходных последовательностей
клавиатурного источника.
Для этого было сформировано~$25$ наборов, 
каждый из которых включал $40$ последовательностей,
полученных различными пользователями на различных компьютерах. 

Было установлено, что ни в одном из наборов наблюдения не повторяются.
Данный факт можно объяснить высокой частотой обновления регистра \texttt{TSC},
несоизмеримой с частотой нажатия оператором на клавиши.
%
Пусть $X_{(1)},X_{(2)},\ldots,X_{(d)}$~--- наблюдения набора~$X$, 
упорядоченные по возрастанию ($d=128\cdot 40$). 
%
В силу неповторяемости $X_{(i)}$
величина~$h=\log_2(X_{(d)}-X_{(1)})$
является адекватной оценкой удельной энтропии на наблюдение для источника
случайности, выдавшего~$X$.
%
Для учета редких длительных пауз между нажатиями на клавиши
при расчетах использовалась уточненная оценка $h^*=\log_2(X_{(3d/4)}-X_{(1)})$,
заведомо меньшая~$h$.

В проведенных экспериментах величина $h^*$ была не меньше $27,1$.
%
Нижняя граница достигалась для процессоров 
с минимально допустимой тактовой частотой $600$~МГц.
%
Полученные результаты дают основание считать, что энтропия выходной
последовательности клавиатурного источника не меньше~$256$,
что достаточно для надежной генерации $256$-битового ключа даже при отказе
физического источника.

%
%Для оценки энтропии был применен метод <<ближайших соседей>>
%(Козаченко Л.~Ф., Леоненко Н.~Н. О статистической оценке энтропии случайного вектора.
%Проблемы передачи информации, 23, 95--101, 1987).
%Пусть $X=(X_1,\ldots,X_n)$~--- набор анализируемых последовательностей
%и пусть последовательности~$X_i$ интерпретируются 
%как векторы $(X_{i1},\ldots,X_{id})$ с вещественными координатами~---
%интервалами между нажатиями клавиш в тактах процессора ($n=20$, $d=128$).
%%
%Оценка энтропии по методу <<ближайших соседей>> имеет вид:
%$$
%H(X)=\frac{d}{n}\sum_{i=1}^n\log_2(R_i)+
%\log_2\left(\frac{(n-1)\pi^{d/2}}{\Gamma(d/2+1)}\right)+\gamma,\quad
%R_i=\min_{1\leq j\leq n,\ j\neq i}\sqrt{\sum_{k=1}^d(X_{ik}-X_{jk})^2}.
%$$
%Здесь $\gamma\approx 0.5772$~-- постоянная Эйлера,
%$\Gamma(z)$~--- гамма-функция Эйлера.
%
%%
%%В проведенных нами экспериментах величина~$H(X)/d$ 
%(удельная энтропия на одно наблюдение) была не меньше~$27.1$. 
%%
%Нижняя граница достигалась для процессоров 
%с минимально допустимой тактовой частотой $600$~МГц.
%%
%Дополнительно выходные наборы преобразовывались в файлы и 
%упаковывались распространенными архиваторами. 
%Коэффициент сжатия не превышал $1/2$, 
%что согласуется с оценкой $H(X)/d$. 
%%
%Было установлено также, что ни в одном из полученных наборов~$X$ 
%наблюдения~$X_{i,k}$ не повторялись.
%Данный факт подтвержает высокую вариабельность данных. 
%%
%%Кроме этого учитывались результаты зарубежных исследователей,
%%согласно которым $2.5$ бита удельной энтропии на наблюдение
%%считается адекватной оценкой даже при замере промежутков времени между 
%%сетевыми событиями (Viega~J., Messier~M. 
%%Secure programming cookbook for C and C++, O'Reilly, 2003).
%

Оценка энтропии физического источника проведена разработчиком 
микроконтроллера токена. 
Установлено, что источник выдает случайные равновероятные независимые слова. 
Поэтому энтропия выходных $256$-битовых строк источника 
оценена максимальным значением~--- $256$.

\paragraph*{Тестирование выходных последовательностей.}
Для проверки работоспособности физического источника случайности
используются статистические тесты американского стандарта 
FIPS PUB 140-2 Security Requirements for Cryptographic Modules.
Тестируется двоичная последовательность длины $20000$.
Тесты имеют следуюший вид:
\begin{enumerate}
\item
{\it Тест знаков}. Определяется величина~$S$~--- 
число единиц в последовательности. 
Тест пройден, если $9725<S<10275$.

\item
{\it Покер-тест}.
Последовательность разбивается на $5000$ тетрад.
Тетрады интерпретируется как числа от $0$ до $15$.
Определяется статистика~$S=16\sum_{i=0}^{15}S_i^2-(5000)^2$,
где~$S_i$~--- количество появлений числа $i$ среди тетрад.
Тест пройден, если $10800<S<230850$.

\item
{\it Тест серий}.
Определяются серии (максимальные последовательности повторяющихся соседних 
битов) различных длин. 
Тест пройден, если и для серий из нулей, и для серий из единиц выполняется: 
$S_1\in[2315,2685]$,
$S_2\in[1114,1386]$,
$S_3\in[527,723]$,
$S_4\in[240,384]$,
$S_5,S_{6+}\in[103,209]$.
Здесь~$S_i$~--- количество серий длины $i=1,2,\ldots$, 
$S_{6+}=S_6+S_7+\ldots$.

\item
{\it Тест длинных серий}.
Тест пройден, если в последовательности отсутствуют серии длины $26$ и больше.
\end{enumerate}

Пороги тестов выбраны так, что вероятность ошибки первого рода (ложной тревоги)
равняется $0,0001$.
Это означает, что даже при нормальной работе микроконтроллера 
в среднем одна из $10000$ его выходных последовательностей
будет забракована.
%
Возможность ложной тревоги должна учитываться при анализе причин блокировки
\CryptoDisk во время тестирования.

\hiddensection{Идентификация и аутентификация}

\paragraph*{Идентификация операторов.}
Идентификация и аутентификация операторов выполняются 
средствами операционной системы и токена.

Администратору назначается идентификатор \texttt{admin}.
Идентификаторы пользователям назначает сам администратор.
Операционная система поддерживает уникальность идентификаторов.
%
Идентификаторы пользователей переписываются на их токены в сервисе~\ref{S.InitToken}.
В сервисе~\ref{S.AuthToken} идентификатор оператора сравнивается 
с идентификатором на токене.

\paragraph*{Аутентификация операторов.}
Для успешной аутентификации требуется предъявить пароль операционной системы,
токен и PIN-код доступа к токену.
%
Пароль и PIN-код вводятся в специальном диалоговом окне. 
Введенные символы маскируются.

Используются сильные пароли Windows, которые состоят не менее чем из $7$ символов,
обязательно содержат буквы в верхнем регистре,
буквы в нижнем регистре,
цифры и
специальные символы (знаки пунктуации, скобки, знаки арифметических операций).
%
PIN-код состоит из $6$ десятичных цифр.

Если при аутентификации операционной системой 
оператор трижды вводит неверный пароль, то компьютер блокируется на $1$ мин.
Для этого администратор настраивает 
политику блокировки учетных записей пользователей
операционной системы.
Если оператор трижды вводит неверный PIN-код, то токен 
блокируется навсегда, его содержимое очищается.

\paragraph*{Качество секретов аутентификации.}
Для проверки качества паролей администратор настраивает 
политику управления паролями операционной системы.

Управляющая программа токена не позволяет задавать PIN-коды,
длина которых отлична от~$6$. Настройка программы не требуется.

\hiddensection{Настройка среды}

\paragraph*{Безопасная установка.}
Перед установкой \CryptoDisk администратор настраивает 
группы операционной системы.
Группа Administrators соответствует роли~\ref{Role.Admins},
группа Users~--- роли~\ref{Role.Users}.

Администратор устанавливает защиту от создания 
неявных копий критических объектов~(Б.2), 
настраивает политику управления паролями и 
политику блокировки учетных записей пользователей (Б.8).
%
Групповая политика настраивается таким образом, 
что установка программ, включая \CryptoDisk, 
разрешена только членам группы Administrators.
%
Дополнительно администратор настраивает средства 
защиты от вредоносных программ.

После этого администратор устанавливает \CryptoDisk.

%Конфигурация пользователя-Административные-шаблоны-
%Компоненты Windows-Консоль управления Microsoft-
%Запрещенные или разрешенные оснастки-
%Групповая политика-
%Расширение оснастки групповой и результирующей политики в свойствах:
%по умолчанию везде "Не задано". 

\paragraph*{Защита системных объектов.}
Файлы программ и настроек, общедоступный справочник открытых 
ключей, флаг блокировки хранятся в специальном каталоге операционной системы.
Пользователям (группа Users) запрещается изменять содержимое этого каталога.

\paragraph*{Защита сеансов.}
Защита сеансовых объектов и аутентификационных данных 
выполняется ядром операционной системы. Настройка не требуется.

\hiddensection{Гарантийные меры}

\paragraph*{Проектирование.}
Разрабатывается документ <<Описание программы>>.
Документ описывает функции, реализующие сервисы~\TOE,
соответствующие структуры данных и коды ошибок.
Описываются состояния программы, форматы хранения объектов.
Устанавливается соответствие с функциональной спецификацией.

Криптографические алгоритмы реализует библиотека \CryptoKernel.
Документ <<Описание программы>> определяет интерфейсы этой библиотеки.

\paragraph*{Разработка.}
Программы разрабатываются на языке~C++ в среде Microsoft Visual Studio 2010. 
Программы библиотеки~\CryptoKernel разрабатываются на языке~C. 

\paragraph*{Управление конфигурацией.}
Для управления исходными текстами программ используется система Subversion.
Система обеспечивает контроль версий модулей программ, 
управление доступом к исходным текстам для нескольких разработчиков,
управление сборкой программ.

Документы снабжаются уникальными идентификаторами по правилам ЕСПД.
Идентификатор включает номер версии документа. 
Номер версии увеличивается при внесении в документ изменений.

Разрабатывается документ <<Поддержка жизненного цикла>>,
в котором описываются правила работы с Subversion 
и правила управления документами.

\paragraph*{Поставка.}
Инсталляционная программа~\CryptoDisk размещается в Интернет 
на сайте разработчика. В инсталляционную программу
по технологии Authenticode добавляется ЭЦП разработчика.
В руководстве администратора описывается 
процесс проверки ЭЦП перед установкой~\CryptoDisk.

\paragraph*{Устранение недостатков.}
Используется система отслеживания ошибок Bugzilla,
которая реализует все необходимые функции по устранению недостатков.
В документе <<Поддержка жизненного цикла>> определяются 
правила работы с Bugzilla.

\paragraph*{Руководства.}
Разрабатываются документы <<Руководство администратора>> 
и <<Руководство пользователя>>. Документы содержат раздел
<<Типичные ошибки>>.

\paragraph*{Программа испытаний.}
Разрабатывается система тестов~\CryptoDisk.
Планы тестирования ориентированы на проверку цепочек выполнения сервисов.
Дополнительно тестируются функции криптографической библиотеки~\CryptoKernel. 

Тесты описываются в документе <<Программа и методика испытаний>>.
Документ включает приложения 
<<Анализ покрытия тестами>> и <<Анализ глубины тестирования>>.

\end{appendix}
