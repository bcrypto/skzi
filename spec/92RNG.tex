\begin{appendix}{Б}{справочное}
{Принципы построения генераторов случайных чисел}
\label{RNG} 

\hiddensection{Назначение}\label{RNG.Intro}

Настоящее приложение дополняет пакет СЧ (\ref{RN})
сведениями о принципах построения генераторов случайных чисел. 
%
Приложение конкретизирует требования пакета, способ их реализации и проверки.

\hiddensection{Схема работы}\label{RNG.Design}

Генерация случайных чисел состоит из следующих этапов.

\begin{enumerate}
\item
Опрос источников случайности,
получение цифровых данных и (или) аналоговых сигналов.

\item
Представление аналоговых сигналов в цифровой форме.

\item
Предварительная обработка данных от источников случайности. 
Состоит в кодировании и сжатии данных и направлена на повышение их 
статистического качества. 

\begin{examples}
\addendum{1~С}жатие фон Неймана заключается в замене пары битов~$01$ на~$0$, 
пары~$10$ на~$1$ и игнорировании пар~$00$ и~$11$. При сжатии устраняется 
возможная неравновероятность битов~$0$ и~$1$ в первоначальной 
последовательности.

\addendum{1~П}усть выходные биты источника принимают значения~$0$ и~$1$ с 
вероятностями $1/2\pm\varepsilon$, $|\varepsilon|<1/2$, независимо от других 
битов.
%
Пусть сжатие состоит в замене непересекающихся $n$-фрагментов битов на их сумму 
по модулю~$2$. 
%
Сумма будет принимать значения~$0$ и~$1$ с вероятностями
$1/2\pm 2^{n-1}\varepsilon^n$. Неравновероятность уменьшается.
\end{examples}

\item
Криптографическая обработка данных от источников случайности. 
\end{enumerate}

Этапы могут пропускаться и объединяться. 

Одновременно с опросом источников случайности может проверяться их 
работоспособность.
%
После предварительной обработки данные от источников случайности 
могут подвергаться статистическому тестированию. 

Криптографическая обработка может сопровождаться шагами инициализации 
генератора и повторной инициализации. На этих шагах могут использоваться 
секретные ключи.

Более подробные сведения о схеме работы генераторов случайных чисел 
можно найти в документе~\cite{AIS31}.

\hiddensection{Оценка энтропии}\label{RNG.Entropy}

Возможны следующие подходы к оценке энтропии источников случайности.

\begin{enumerate}
\item 
Составить подробную физическ\addendum{ую} модель источника 
случайности и провести вероятностный анализ модели. Оценка энтропии 
является результатом теоретических выкладок, возможно подкрепленных 
результатами экспериментов. 

\begin{example*}
\addendum{Д}анный способ был применен в работе~\cite{DHM07} для оценки вероятностей
выпадения граней монеты. 
%
В рамках предложенной физической модели, адекватность
которой была подтверждена экспериментами, установлено, что монета упадет на
ту же грань, с которой она была брошена, с вероятностью~$\approx 0.51$.
%
Полученная оценка вероятности может быть трансформирована в оценку энтропии
источника случайности <<монета>>.
\end{example*}

\item 
Составить примерную физическую модель источника, 
обработать выходные данные от источника и по этим данным построить
статистические оценки энтропии. 

\begin{example*}
\addendum{В} клавиатурном источнике при нажатии оператором кнопок клавиатуры 
фиксируются значения регистра~TSC. Это $64$-разрядный регистр-таймер (time 
stamp counter), содержимое которого увеличивается на каждом такте работы 
процессора.
%
% Частота процессора не ниже $600$~MГц.
%
64-битовая разность между значениями регистра сохраняется, если друг за другом 
нажаты две различные клавиши и интервал между нажатиями более $50$~мс. 
Всего сохраняется $128$ разностей. Из них составляется $1024$-байтовая 
выходная последовательность.

Для оценки энтропии было сформировано~$25$ наборов, 
каждый из которых включал $40$ последовательностей.
%
Наборы формировались различными пользователями на различных компьютерах. 
%
Было установлено, что ни в одном из наборов наблюдения не повторяются.
Данный факт можно объяснить высокой частотой обновления регистра TSC,
несоизмеримой с частотой нажатия оператором на клавиши.
%
Пусть $X_{(1)},X_{(2)},\ldots,X_{(d)}$~--- наблюдения набора~$X$, 
упорядоченные по возрастанию ($d=128\cdot 40$). 
%
В силу неповторяемости $X_{(i)}$ величина~$h=\log_2(X_{(d)}-X_{(1)})$
является адекватной оценкой снизу удельной энтропии на наблюдение для 
источника случайности, выдавшего~$X$.
%
Для учета редких длительных пауз между нажатиями на клавиши
при расчетах использовалась уточненная оценка $h^*=\log_2(X_{(3d/4)}-X_{(1)})$,
заведомо меньшая~$h$.

В проведенных экспериментах величина $h^*$ была не меньше $27,1$.
%
Нижняя граница достигалась для процессоров 
с минимально допустимой тактовой частотой $600$~МГц.
\end{example*}

\item 
Использовать известные оценки для распространенных источников случайности.
\end{enumerate}

Использованные при оценке энтропии физические модели источников случайности 
должны быть сложными настолько, что противник, даже располагая большими 
вычислительными ресурсами, не может прогнозировать данные от источников. 

\begin{example}
Пусть источником случайности является таймер. 
Физическая модель: таймер обновляется с высокой частотой, обращения к 
таймеру происходят асинхронно в различных местах программы, 
следовательно, отсчеты таймера в моменты обращения непредсказуемы. 
Тем не менее, противник может составить типовые маршруты выполнения 
программы, получить адекватные оценки задержек между обращениями к 
таймеру и, таким образом, снизить неопределенность данных от 
источников случайности. Вывод: модель не является достаточно сложной; 
источник случайности <<таймер>>, описываемый данной моделью, должен быть 
забракован.
\end{example}

При статистическом оценивании данные от источников (выборки)
обрабатываются функциями (статистиками) из определенного перечня. 
Каждая статистика дает собственную оценку энтропии, среди этих оценок 
выбирается минимальная. Рекомендуется использовать статистики, установленные 
\addendum{в}~\cite{SP800-90B}.  

Перечень статистик зависит от выполнения гипотезы о том, что выходные 
данные источника случайности представляют собой реализации 
независимых одинаково распределенных величин. Гипотеза проверяется 
с помощью специальных статистических тестов.
%
Если гипотеза выполнена, то достаточно использовать всего одну статистику,
основанную на учете самого частого значения в выборке. Данная статистика определена 
\addendum{в~\cite{SP800-90B} (пункт~6.3.1)}.

\hiddensection{Статистические тесты}\label{RNG.StatTest}

Статистический тест проверяет гипотезу~$H_0$ о вероятностной структуре 
источника случайности.
%
Тестирование проводится в процессе эксплуатации источника в составе 
генератора случайных чисел или во время разработки генератора.
%
В последнем случае тестирование \addendum{проводится} более глубоко и подробно. 
%
Чаще всего проверяется гипотеза~$H_0$ об идеальности:
выходные символы источника являются реализациями независимых случайных величин  
с равномерным распределением на выходном алфавите.

Статистический тест обрабатывает выборку выходных символов, 
вычисляет статистику~$S$ и проверяет, что она попадает
в допустимую область~$\Sigma$. При $S\in\Delta$ гипотеза~$H_0$ принимается,
при $S\notin\Delta$~--- отвергается. Область~$\Delta$ обычно представляет
собой интервал на числовой прямой: двусторонний или односторонний 
(открытый справа или слева). Статистика~$S$ может быть многомерной,
и тогда интервалы вводятся для каждой из ее координат.

\begin{example}
В~\cite{FIPS140-2} определены 4 теста. Они применяются к двоичной 
последовательности длины $20000$ и имеют следующий вид.

%\hspace{0.8cm}
1.~Тест знаков. Определяется величина~$S$~--- 
число единиц в последовательности. 
Тест пройден, если $9725<S<10275$.

%\hspace{0.8cm}
2.~Покер-тест. Последовательность разбивается на $5000$ тетрад.
Тетрады интерпретиру\addendum{ю}тся как числа от $0$ до $15$.
Определяется статистика~$S=16\sum_{i=0}^{15}S_i^2-(5000)^2$,
где~$S_i$~--- количество появлений числа $i$ среди тетрад.
Тест пройден, если $10800<S<230850$.

%\hspace{0.8cm}
3.~Тест серий. Определяются серии (максимальные последовательности 
повторяющихся соседних битов) различных длин. 
Тест пройден, если и для серий из нулей, и для серий из единиц выполняется: 
$S_1\in[2315,2685]$,
$S_2\in[1114,1386]$,
$S_3\in[527,723]$,
$S_4\in[240,384]$,
$S_5,S_{6+}\in[103,209]$.
Здесь~$S_i$~--- количество серий длины $i=1,2,\ldots$, 
$S_{6+}=S_6+S_7+\ldots$.

%\hspace{0.8cm}
4.~Тест длинных серий. Тест пройден, если в последовательности отсутствуют 
серии длины $26$ и больше.
\end{example}

Технически статистика~$S$ представляет собой случайную величину, 
которая при выполнении~$H_0$ попадает в область~$\Delta$ 
с вероятностью~$1-\alpha$, где~$\alpha$ невелико.
%
Величина~$\alpha$ называется уровнем значимости теста. 
%
Она определяет вероятность <<ложной тревоги>> (или ошибки первого рода): 
отвергнуть~$H_0$ при ее соблюдении.

Статистику~$S$ можно преобразовать в~$P$-значение~$P$, 
которое при соблюдении~$H_0$ будет иметь равномерное распределение на 
отрезке~$[0,1]$. Фактически $P$-значение представляет собой унифицированную 
статистику. При работе с~$P$-значениями в качестве~$\Delta$ обычно выбирается
отрезок~$(\alpha,1]$, т.~е. гипотеза~$H_0$ отвергается, если~$P\leq\alpha$.

\begin{example}
Интервалы статистических тестов, описанных в предыдущем примере,
выбраны так, что уровень значимости $\alpha=0,0001$.
%
Это означает, что при нормальной работе тестируемого источника
в среднем одна из $10000$ его выходных последовательностей
будет забракована отдельным тестом. 

Если применяется единственный тест,
тестирование проводится 1000 раз в год, 
то в течение 5 лет эксплуатации <<ложная тревога>>
случится хотя бы один раз с вероятностью
$$
1-(1-\alpha)^{5000}\approx 0,39.
$$
\end{example}

Для повышения качества тестирования может использоваться~$N$ 
тестов. Они могут обрабатывать $M$ выборок. В результате
тестирования формируется матрица $(B_{ij})$, в которой $B_{ij}=0$,
если гипотеза~$H_0$ принята $i$-м тестом на $j$-й выборке,
и $B_{ij}=1$, если~$H_0$ отвергнута.
%
Матрица~$(B_{ij})$ обрабатывается блоком окончательного решения. 
Набор базовых тестов вместе с этим блоком называется батареей.

\begin{example}
Продолжим предыдущий пример. Пусть к выборке применяются 
все 4 теста. Гипотеза $H_0$ отклоняется, если ее отвергают 
два и более теста. Это произойдет с вероятностью
$$
\beta=6\alpha^2 + 4 \alpha^3 + \alpha^4.
$$
Здесь предполагается статистическая независимость тестов.
Она подтверждается теоретическими и экспериментальными 
исследованиями.

При новой стратегии тестирования <<ложная тревога>>
случится в течение 5 лет с вероятностью
$$
1-(1-\beta)^{5000}\approx 0,0003.
$$

Еще раз изменим стратегию. Пусть обрабатывается 2 выборки.
Гипотеза $H_0$ отклоняется, если какую-то из выборок отвергают 
два и более теста или один из тестов отвергает $H_0$ дважды.
%
Вероятность <<ложной тревоги>>:
$$
1-(1-2\beta+\beta^2-4\alpha^2)^{5000}\approx 0,0008.
$$
\end{example}

В более сложных батареях, которые обычно используются при исследовании
генераторов, вместо признаков $B_{ij}$ используются $P$-значения~$P_{ij}$.
%
Блок окончательного решения проверяет независимость $P$-значений
и их равномерное распределение на отрезке $[0,1]$.

\end{appendix}
