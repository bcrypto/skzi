\section{Уровни безопасности}\label{COMMON.Levels}

При соблюдении требований определенного уровня безопасности обеспечивается
защита от противника (нарушителя) с определенным потенциалом.
%
Противник находится в среде эксплуатации, 
располагает образцами СКЗИ, 
всеми руководствами и программами, \addendum{относящимися к СКЗИ}. 
%
Противник имеет логический и штатный физический доступ к СКЗИ: 
выполняет доступные сервисы, 
делает попытки аутентификации,
подключает СКЗИ к штатным коммуникационным устройствам, 
подключает к CКЗИ штатные носители информации и др. 
%
Максимальный дополнительный потенциал противника на каждом из уровней 
представлен в таблице~\ref{Table.COMMON.Levels} и объясняется ниже.

Противник может иметь нештатный физический доступ к СКЗИ.
Например, он может подключать нештатные устройства и носители или 
делать попытки вскрытия корпуса.

Квалификация противника может быть базовой, средней или высокой.
%
Противник базовой квалификации проводит эксперименты с образцами СКЗИ, 
изучает руководства. 
%
Средняя квалификация дополнительно предполагает анализ программного обеспечения 
СКЗИ и КСК. При анализе могут использоваться автоматизированные средства поиска 
ошибок в исходных текстах программ, сканеры уязвимостей, сведения об известных 
уязвимостях.
%
Наконец противник высокой квалификации дополнительно проводит исследование
аппаратных КСК и аппаратных компонентов СКЗИ. Изучается поведение 
компонентов при внешних воздействиях, ведется поиск уязвимостей, 
снова используются сведения об известных уязвимостях.

% Варианты: 
% - детализировать "средняя" -- готовые программы (общеизвестные сведения) 
%     vs собственные программы
% - разрушающие / неразрушающие действия

Для организации атак противник может использовать вредоносные программы:
перехватчики клавиатуры (кейлогеры), шпионские программы, 
средства перебора паролей и др. 
%
Противник может адаптировать уже готовые программы или разработать собственные. 
%
Вредоносные программы могут быть универсальными или ориентированными на 
конкретное СКЗИ.

В атаках могут также использоваться \addendum{технические} средства: 
средства перехвата сигналов в побочных каналах,
средства генерации физических воздействий,
\addendum{средства перехвата трафика},
оборудование для исследовани\addendum{я} аппаратных компонентов.

% todo: убрать оборудование?

\begin{table}[hbt]
\caption{Потенциал противника}\label{Table.COMMON.Levels}
\begin{tabular}{|c|c|c|c|c|}
\hline
\multirow{2}{*}{Уровень} 
& Нештатный & \multirow{2}{*}{Квалификация} & Вредоносные & \addendum{Технические}\\
& физический доступ & & программы & средства\\
\hline
\hline
1 & $-$ & базовая & $-$ & $-$\\
2 & $-$ & средняя & $+$ & $-$\\
3 & $+$ & средняя & $+$ & $-$\\
4 & $+$ & высокая & $+$ & $+$\\
\hline
\end{tabular}
\end{table}

Для защиты от атак противника для СКЗИ уровня 1 предусматривается логическая 
защита криптографической границы (через аутентификацию).
%
В системной среде организуется управление доступом, ведется контроль КСК.
% 
%При этом системная среда может быть модифицируемой.
%
Требуется, чтобы собственные компоненты СКЗИ были надежно спроектированы,
достаточно полно проанализированы и протестированы.

Типовое СКЗИ уровня 1: программное средство, установленное на персональном
компьютере под управлением \addendum{универсальной} операционной системы с 
обычными (по умолчанию) настройками безопасности.

На уровне 2 \addendum{о}граничивается модифицируемость системной среды~-- 
разрешается устанавливать только доверенные программы.
%
% todo: доверенные?
%
Вводится отслеживание уязвимостей системных компонентов.
%
Повышается надежность собственных компонентов СКЗИ, 
%
% в частности,
% требуется обеспечить защиту от создания неявных копий критических объектов, 
% независимость времени выполнения криптографических алгоритмов от значений 
% обрабатываемых критических объектов. 
%
усиливаются гарантии качества компонентов.
%
% - детализация внутренних интерфейсов;
% - контроль инсталяционных программ;
% - cистема устранения недостатков;
% - анализ типичных ошибок операторов;
% - полное тестирование сервисов СКЗИ.

Примеры СКЗИ уровня 2: 
программное обеспечение удостоверяющего центра или другой криптографической службы;
%
программное средство, загружаемое вместе с операционной системой с 
фиксированного установочного носителя.
% 
% todo: прикладная программа (приложение) для смартфона?
% stateless operating systems: Incognito, Tails
%
Уровня 2 могут достигать программные СКЗИ, выполняемые на облачных серверах.

На уровне 3 вводится пассивная физическая защита криптографической границы,
появляется контроль интерфейсов взаимодействия с устройствами за пределами 
границы. Системная среда становится немодифицируемой~--- разрешается только 
обновление программ СКЗИ. Вводится аудит событий безопасности в системной среде.

Примеры СКЗИ уровня 3: криптографический USB-токен, ID-карта, 
аппаратный IP-шифратор под управлением массовой операционной системы.

На уровне 4 появляются активная физическая защита криптографической границы,
защита от внешних воздействий, 
контроль физических побочных каналов.
%
\addendum{Усиливается логическая защита границы~--- требуется 2-факторная 
аутентификация.}
%
Программное обеспечение КСК, в том числе операционная система, становится 
частью СКЗИ и подпадает под полномасштабный анализ исходных текстов и 
тестирование. 
%
% полный анализ исходных текстов (?).

Пример СКЗИ уровня 4: аппаратное средство 
\addendum{cо специализированной} операционной системой собственной разработки
и усиленной физической защитой. 

