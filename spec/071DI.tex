% Design and Implementation
\section{Проектирование и разработка (ПР)}\label{DI}

\subsection{Обзор}\label{DI.Intro}

Пакет ПР устанавливает требования по проектированию и разработке 
\addendum{сервисов и механизмов безопасности} СКЗИ
\addendum{в соответствии с} функциональной спецификаци\addendum{ей}.
%
\addendum{Устройство (структура) и способ реализации сервисов и механизмов 
детализируются в описаниях программ и аппаратных компонентов СКЗИ}.
%
Программы и компоненты разрабатываются \addendum{по описаниям или вместе с 
описаниями}. 

\addendum{
Детализация сервисов включает спецификацию их интерфейсов, которые 
выступают в роли внешних интерфейсов СКЗИ~\forref{R.DI.HLD}.}

\addendum{
Детализация механизмов безопасности включает выделение и спецификацию 
внутренних компонентов СКЗИ, реализующих механизмы и вспомогательные 
функциональные блоки.
%
Примеры внутренних компонентов: 
библиотека программных реализаций криптографических алгоритмов (криптоядро),
генератор случайных чисел~\forref{R.DI.LLD}.
}

\addendum{Пакет ПР не устанавливает правила оформления описаний программ и 
аппаратных компонентов. Правила могут быть определены за пределами настоящего 
стандарта. Например, может требоваться оформлять описания в виде 
комплектов программной и (или) конструкторской документации 
в соответствии с номенклатурами ГОСТ 19.101 и (или) ГОСТ 2.102.
}

\subsection{Требования}\label{DI.Reqs}

\req{ПР}{1--4}\label{R.DI.Spec} % 1
\addendum{
Должны быть разработаны описания программ и аппаратных компонентов~СКЗИ.
Описания должны соответствовать функциональной спецификации.
}

\req{ПР}{1--4}\label{R.DI.HLD} % 2
В описани\addendum{ях}~\useref{R.DI.Spec}
должны быть определен\addendum{ы в}нешние интерфейсы~СКЗИ.

\req{ПР}{2--4}\label{R.DI.LLD} % 3
В описани\addendum{ях}~\useref{R.DI.Spec}
должны быть определены внутренние компоненты~СКЗИ и их интерфейсы.

\req{ПР}{1--4}\label{R.DI.Tools} % 4
В описани\addendum{ях}~\useref{R.DI.Spec} должны быть 
определен\addendum{ы и}спользуемые средства разработки и сборки программ. 
Должны быть перечислен\addendum{ы к}онфигурационные файлы, отвечающие за 
настройку средств разработки и сборки\addendum{.}

\req{ПР}{1--4}\label{R.DI.Comments} % 5
Исходные тексты программ должны быть снабжены комментариями,
\addendum{отражающими связь программ} с описани\addendum{ями}~\useref{R.DI.Spec}.

\req{ПР}{1--4}\label{R.DI.Language} % 6
Программы должны быть написаны на высокоуровневых языках программирования.
Вставки на низкоуровневых языках (языках ассемблера) допускаются в случаях,
критичных для производительности, а также тогда, когда высокоуровневые языки
применить нельзя.

\begin{note}
\addendum{
Низкоуровневые языки предназначены для прямой работы с инструкциями процессора. 
Напротив, в высокоуровневых языках архитектура процессора абстрагируется, 
используются конструкции, удобные программистам и поэтому  
лучше читаемые, более управляемые и менее подверженные ошибкам.}
\end{note}

\req{ПР}{4}\label{R.DI.OS} % 7
\addendum{В состав СКЗИ должно быть~\forref{R.DI.Spec} включено программное 
обеспечение КСК}.

\begin{note}
Включение \addendum{программного обеспечения} в состав СКЗИ означает 
\addendum{взятие его под контроль}~--- тестирование~\forref{R.TE.Deep}
и анализ исходных текстов~\forref{R.CR.Detailed}. 
%
\addendum{
Операционная система, если она имеется, также подпадает под контроль. 
%
Речь может идти о специализированной системе (системном программном обеспечении) 
собственной разработки, об облегченной универсальной (встраиваемой, 
embedded) системе с открытым кодом, а также о других вариантах исполнения,
допускающих тестирование и анализ исходных текстов.}
\end{note}

