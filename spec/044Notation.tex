\section{Оформление требований}\label{COMMON.Notation}

Требования безопасности вводятся пакетно. В преамбуле пакета определяется 
его назначение, дается обзор понятий, необходимых для формулировки 
требований пакета и правильной их интерпретации.
%
Сквозной обзор представлен в~\ref{COMMON.Overview}.

Требования внутри пакета нумеруются последовательно, начиная с~единицы.
%
Номер требования через точку присоединяется к коду пакета, 
в результате получается полное имя требования: КП.1, РС.2 и т.д.

Для каждого требования в круглых скобках перечисляются уровни, 
на которых требование выдвигается.
Примеры: (4), (1--4), (3, 4).

В тексте имеются ссылки на требования. 
%
Для ссылки на сущности, введенные в требовании T, используется 
форма [T] (прямая ссылка).
%
Для детализации T с учетом условий и пояснений в месте ссылки 
используется  форма \{T\} (обратная ссылка).

Формулировки требований безопасности могут включать указания на 
необходимость определения списка методов, списка компонентов, 
списка механизмов и др.
%
Если не оговорено противное, то результатом определения 
может быть пустой список.
