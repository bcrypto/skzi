% Non-Invasive security
\section{Защита от утечек (ЗУ)}\label{NI}

\subsection{Обзор}\label{NI.Intro}

Пакет ЗУ устанавливает требования по защите \addendum{СКЗИ от} атак, 
использующих утечку информации по побочным каналам и не требующих прямого 
физического контакта с СКЗИ.

Атаки проводит противник высокой квалификации, оснащенный 
\addendum{техническими} средствами. 
\addendum{Атаки основываются на анализе сигналов в побочных каналах.
Примеры сигналов: 
мощность, потребляемая при выполнении криптографических операций,   
электромагнитные излучения и наводки~\forref{R.NI.Channels}}.

Защита от утечки информации по побочным каналам \addendum{обеспечивается} 
за счет технических решений, принятых при проектировании и реализации 
СКЗИ. \addendum{
Допускается наличие каналов, для защиты от утечек по которым дополнительно 
применяются внешние технические средства (генератор электромагнитного шума)
и организационные меры (установка контролируемой 
зоны)~\forref{R.NI.Protect,R.NI.External}}.

Эффективность механизма защиты характеризуется количественной метрикой.
Метрика может определять степень ослабления сигнала, энергию сигнала, 
дальность передачи.
%
\addendum{
Допускается не использовать количественные метрики и характеризовать 
эффективность механизма защиты качественно. Например, 
<<потребляемая мощность не содержит информацию о критических объектах>>.
%
Рекомендуется использовать количественные метрики эффективности~\forref{R.NI.Protect}.
} 

\subsection{Требования}\label{NI.Reqs}

% каналы

\req{ЗУ}{4}\label{R.NI.Channels} % 1
Должен быть определен перечень побочных каналов. 

% механизмы защиты

\req{ЗУ}{4}\label{R.NI.Protect} % 2
\addendum{
Для каждого побочного канала~\useref{R.NI.Channels}
должны быть определены и корректно реализованы механизмы защиты 
критических объектов СКЗИ от утечки по каналу. 
%
Для каждого механизма должен быть указана метрика эффективности,
количественная или качественная.
}

% измерения для измеримых методов

\req{ЗУ}{4}\label{R.NI.Estimate} % 3
Для каждого \addendum{механизма защиты~\useref{R.NI.Protect}}, для которого 
определены количественные метрики эффективности, должна быть проведена 
экспериментальная оценка соответствия СКЗИ метрик\addendum{ам}.

% измерения для измеримых методов

\req{ЗУ}{4}\label{R.NI.External} % 4
\addendum{
Должны быть определены побочные каналы~\useref{R.NI.Channels},
для которых используемые механизмы защиты~\useref{R.NI.Protect}
недостаточно эффективны. Для этих каналов должны быть 
определены~\forref{R.GD.Admin,R.GD.Roles} внешние технические 
средства и (или) организационные меры защиты от утечек. 
}
