% Non-Invasive security
\section{Защита от утечек (ЗУ)}\label{NI}

\subsection{Обзор}\label{NI.Intro}

Пакет ЗУ устанавливает требования по защите \addendum{СКЗИ от} атак, 
использующих утечку информации по побочным каналам и не требующих прямого 
физического контакта с СКЗИ.

Атаки проводит противник высокой квалификации, оснащенный 
\addendum{техническими} средствами. 
\addendum{Атаки основываются на анализе сигналов в побочных каналах.
Примеры сигналов: 
потребляемая мощность при выполнении операций в СКЗИ (simple power analysis),   
изменения потребляемой мощности (differential power analysis), 
электромагнитные излучения и наводки~\forref{R.NI.Channels}}.

Защита от утечки информации по побочным каналам \addendum{обеспечивается} 
за счет технических решений, принятых при проектировании и реализации 
СКЗИ~\addendum{\forref{R.NI.Protect}}. 
%
% \addendum{а также применением внешних средств, например, генераторов 
% шума~\forref{R.NI.Protect}}.

Эффективность механизма защиты характеризуется количественной метрикой.
Метрика может определять степень ослабления сигнала, \doubt{энергию сигнала}, 
\doubt{дальность передачи}.
%
\addendum{
Допускается не использовать количественные метрики и характеризовать 
эффективность механизма защиты качественно. Например, 
\doubt{<<экран не пропускает электромагнитные излучения>>}.
%
Рекомендуется использовать количественные метрики эффективности.
} 

\if0
Механизмы защиты могут быть качественными. Например: 
<<В СКЗИ применено экранирование активных элементов для защиты от побочных 
электромагнитных излучений>>. 
%
Для повышения гарантий защиты рекомендуется приводить количественные метрики 
оценки эффективности механизмов. Например, можно указать радиус зоны, 
за пределами которой исключена возможность регистрации опасных сигналов 
СКЗИ противником.
\fi

\subsection{Требования}\label{NI.Reqs}

% каналы

\req{ЗУ}{4}\label{R.NI.Channels} % 1
Должен быть определен перечень побочных каналов. 

Для каждого канала из перечня
долж\addendum{ны} быть указа\addendum{ны} примененны\addendum{е} 
механиз\addendum{мы} защиты и дана оценка \addendum{их} эффективности.
                             	
% механизмы защиты

\req{ЗУ}{4}\label{R.NI.Protect} % 2
\addendum{
Для каждого побочного канала~\useref{R.NI.Channels}
должны быть определены и корректно реализованы механизмы защиты 
критических объектов СКЗИ от утечки по каналу. 
%
Для каждого механизма должен быть указана метрика эффективности,
количественная или качественная.
}

% измерения для измеримых методов

\req{ЗУ}{4}\label{R.NI.Estimate} % 3
Для каждого \addendum{механизма защиты~\useref{R.NI.Protect}}, для которого 
определены количественные метрики эффективности, должна быть проведена 
экспериментальная оценка соответствия СКЗИ метрик\addendum{ам}.

