% Non-Invasive security
\section{Защита от утечек (ЗУ)}\label{NI}

\subsection{Обзор}\label{NI.Intro}

Пакет ЗУ устанавливает требования по защите аппаратного СКЗИ от несанкционированного 
доступа к критическим объектам СКЗИ путем пассивных атак, использующих утечку 
информации по побочным каналам и не требующих прямого физического контакта с 
СКЗИ. 

Атаки проводит противник высокой квалификации, оснащенный 
\addendum{техническими} средствами. Атаки могут основываться на измерении 
потребляемой  мощности при выполнении операций в СКЗИ (simple power analysis), 
анализе изменения потребляемой мощности при выполнении операций в СКЗИ 
(differential power analysis), съеме и анализе электромагнитных излучений и 
наводок при функционировании СКЗИ.

Защита от утечки информации по побочным каналам может быть обеспечена в СКЗИ за 
счет технических решений, принятых при проектировании и реализации СКЗИ. 
%
Механизмы защиты могут быть качественными. Например: 
<<В СКЗИ применено экранирование активных элементов для защиты от побочных 
электромагнитных излучений>>. 
%
Для повышения гарантий защиты рекомендуется приводить количественные метрики 
оценки эффективности механизмов. Например, можно указать радиус зоны, 
за пределами которой исключена возможность регистрации опасных сигналов 
СКЗИ противником.

\subsection{Требования}\label{NI.Reqs}

% каналы

\req{ЗУ}{4}\label{R.NI.Channels}
Должен быть определен перечень побочных каналов. Для каждого канала из перечня
должен быть указан примененный в СКЗИ механиз\addendum{м} защиты и дана оценка его
эффективности.
                             	
% методы защиты

\req{ЗУ}{4}\label{R.NI.Protect}
Должны быть определены и корректно реализованы механизмы защиты критических 
объектов СКЗИ от раскрытия за счет утечки информации по побочным 
каналам~\useref{R.NI.Channels}. 

% измерения для измеримых методов

\req{ЗУ}{4}\label{R.NI.Estimate}
Для каждого побочного канала~\useref{R.NI.Channels}, для которого определена 
количественная метрика эффективности примененного механизма защиты, 
должна быть проведена экспериментальная оценка соответствия СКЗИ 
метрике.

