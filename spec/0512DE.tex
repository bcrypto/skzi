% Decommissioning
\section{Вывод из эксплуатации (ВЭ)}\label{DE}

\subsection{Обзор}\label{DE.Intro}

Пакет ВЭ устанавливает требования к очистке объектов СКЗИ перед
выводом его из эксплуатации. Очистка блокирует работу со средством от  
лица его штатного оператора. Очистка может дополнительно препятствовать 
определению личности оператора. Очистку выполняет сам оператор,
возможно администратор.

Достаточно очистить только те критические объекты, которые хранятся в 
пределах криптографической границы в открытом виде.
%
Например, очистить ключ защиты ключей, но не ключи парной связи, на нем 
зашифрованные.
%
Однако существует угроза раскрытия защищенных ключей, которые могут 
циркулировать за пределами СКЗИ. Для повышения гарантий безопасности требуется 
очищать все критические объекты~\forref{R.DE.Short,R.DE.Long}.

Кроме очистки объекта, может потребоваться его удаление. Например, когда ключ 
размещен в файле, имя которого совпадает с именем оператора, и при этом нужно 
скрыть личность оператора.

%Радикальной формой вывода из эксплуатации является возврат к заводским 
%настройкам, при котором очищаются и удаляются все объекты, созданные в процессе 
%функционирования СКЗИ. 

\subsection{Требования}\label{DE.Reqs}

\req{ВЭ}{1--4}\label{R.DE.List}
Должен быть определен перечень критических объектов, 
после очистки которых работа с СКЗИ от имени оператора
той или иной роли станет невозможной. 
%
В перечень могут быть включены объекты, которые указывают на оператора.

\req{ВЭ}{1, 2}\label{R.DE.Short}
В перечень~\useref{R.DE.List} должны быть включены все критические объекты, 
которые хранятся в пределах криптографической границы в открытом виде.

%Даже если объект хранится в открытом виде, по правилам пакета ЗО он защищен 
%либо аппаратно, либо организационно (это частичный секрет).

\req{ВЭ}{3, 4}\label{R.DE.Long}
В перечень~\useref{R.DE.List} должны быть включены все критические объекты, 
которые хранятся в пределах криптографической границы.

\req{ВЭ}{1--4}\label{R.DE.Service}
Должен быть~\forref{R.SV.List} предусмотрен сервис принудительной 
очистки~\useref{R.DP.Zero} всех объектов перечня~\useref{R.DE.List}.
%
Сервис должен быть~\forref{R.SV.List} критическим.
%
Сервис должен быть доступен~\forref{R.AC.Policy} оператору~--- владельцу объектов,
возможно администратору.
%
Если сервис вызывается в сеансе владельца, то этот сеанс должен быть 
завершен сразу после очистки.

\req{ВЭ}{4}\label{R.DE.Fast}
Сервис принудительной очистки~\useref{R.DE.Service} должен выполняться 
без задержки и за достаточно малый промежуток времени.

\req{ВЭ}{3, 4}\label{R.DE.AU}
Принудительная очистка должна быть~\forref{R.AU.Events} включена в перечень 
событий аудита.

