\begin{appendix}{А}{рекомендуемое}
{Содержание функциональной спецификации}\label{SPEC}

\mbox{}

\begin{enumerate}
\item
{Описание~СКЗИ.}

\begin{enumerate}
\item
Назначение.
\item
Уровень ($1$, $2$, $3$ или $4$).
\item
Основные функциональные возможности.
\item
Криптографическая граница.
\item
Список критических системных компонентов~\useref{R.ST.CSCList}.
\end{enumerate}

\item
{Криптографическая поддержка.}

\begin{enumerate}
\item
Список криптографических алгоритмов~\useref{R.CS.Algs}.

\item
Методы генерации долговременных параметров и ключей~\useref{R.CS.Gen}.

\item
Механизмы контроля времени выполнения криптографических 
алгоритмов~\useref{R.CS.Timing} (начиная с уровня 2). 
\end{enumerate}

\item
{Реализация сервисов.}
\begin{enumerate}
\item
Список сервисов~\useref{R.SV.List}.

\item
Механизмы защиты от создания неявных копий
критических объектов~\useref{R.SV.Leaks} 
(начиная с уровня 2).
\end{enumerate}

\item
{Управление доступом.}

\begin{enumerate}
\item
Список ролей операторов~\useref{R.AC.Roles}.

\item
Список объектов~\useref{R.AC.Objects}.

\item
Описание политики управления доступом~\useref{R.AC.Policy}.

\item
Состояния СКЗИ и правила перехода между состояниями~\useref{R.AC.States}.
\end{enumerate}

\item
{Защита объектов.}

\begin{enumerate}
\item
Криптографические методы обеспечения конфиденциальности~\useref{R.DP.CryptoE}.

\item
Криптографические методы контроля целостности и 
подлинности~\useref{R.DP.CryptoI}. 

\item
Аппаратные методы защиты~\useref{R.DP.Hard}.

\item
Методы разделения секрета~\useref{R.DP.Split}.

\item
Алгоритмические методы контроля целостности~\useref{R.DP.Algo}.

\item
Организационные меры обеспечения конфиденциальности~\useref{R.DP.Org}.

\item
Соответствие <<объекты~--- методы защиты>>~\useref{R.DP.Crit,R.DP.Public}.

\item
Методы очистки критических объектов~\useref{R.DP.Zero}.

\item
Системные методы защиты~\useref{R.DP.Sys} (уровень 1).
\end{enumerate}

\item
{Самотестирование.}

\begin{enumerate}
\item
Проверка работоспособности критических системных компонентов~\useref{R.ST.CSCTests}.

\item
Перечень проверок самотестирования~\useref{R.ST.Tests}.

\item
Обработка ошибок тестирования~\useref{R.ST.TestLock}.
\end{enumerate}

\item
Аудит (начиная с уровня 3).

\begin{enumerate}
\item
Перечень регистрируемых событий~\useref{R.AU.Events}.

\item
Структура записей аудита~\useref{R.AU.Records}.

\item
Обработка переполнения журнала аудита~\useref{R.AU.Over}.
\end{enumerate}

\item
Физическая безопасность (начиная с уровня 3).

\begin{enumerate}
\item
Механизмы физической защиты~\useref{R.PS.List}.

\item
Действия при срабатывании механизмов контроля 
доступа~\useref{R.PS.Erasing} (уровень 4). 
\end{enumerate}

\item
Защита от воздействий (уровень 3).

\begin{enumerate}
\item
Допустимые границы температуры, напряжения и др.~\useref{R.EF.Ranges}.

\item
Механизмы обнаружения выхода за допустимые границы~\useref{R.EF.Detect}
(уровень 4).
\end{enumerate}

\item
Защита от утечек (уровень 4).

\begin{enumerate}
\item
Перечень побочных каналов~\useref{R.NI.Channels}.

\item
Механизмы защиты от утечек~\useref{R.NI.Protect}.

\item
Экспериментальные оценки утечек~\useref{R.NI.Estimate}.
\end{enumerate}

\item
Генерация случайных чисел (при включении пакета СЧ).

\begin{enumerate}
\item
Описание генераторов случайных чисел~\useref{R.RN.Spec}.

\item
Оценка энтропии источников случайности~\useref{R.RN.Entropy}.

\item
Проверка работоспособности~\useref{R.RN.TotTest}.

\item
Статистическое тестирование~\useref{R.RN.Tests}.
\end{enumerate}

\item
Обновление программ (при включении пакета ОП).

\begin{enumerate}
\item
Контрол\addendum{ь} целостности и подлинности обновляемых 
программ~\useref{R.SU.Import}.

\item
Проверка хранения корректных программных модулей~\useref{R.SU.PoS} (уровень 4).
\end{enumerate}

\item
Вывод из эксплуатации (при включении пакета ВЭ).

\begin{enumerate}
\item
Перечень объектов, подлежащих очистке~\useref{R.DE.List}.
\end{enumerate}

\item
{Идентификация и аутентификация.}

\begin{enumerate}
\item
Методы идентификации операторов~\useref{R.IA.Id}.

\item
Методы аутентификации операторов~\useref{R.IA.AuthData,R.IA.Auth}.

\item
Проверка качества секретов аутентификации~\useref{R.IA.PwdSet}.
\end{enumerate}

\item
{Настройка среды.}

\begin{enumerate}
\item
Настройка среды для безопасной установки~\useref{R.ES.Install}.

\item
Настройка среды для защиты системных объектов~\useref{R.ES.Objects}.

\item
Настройка среды для защиты сеансов~\useref{R.ES.Session}.

\item
Отслеживание уязвимостей КСК~\useref{R.ES.CVE} (начиная с уровня 2).

\end{enumerate}

\item
Доверенный канал (при включении пакета ДК).

\begin{enumerate}
\item
Встречная аутентификация~\useref{R.TC.Auth2}.
\item
Защита канала~\useref{R.TC.Crypto}.
\end{enumerate}


\item
Гарантийные меры (могут быть определены в отдельных документах).
\begin{enumerate}
\item
Описания программ и аппаратных компонентов~\useref{R.DI.Spec}.

\item
Внешние интерфейсы~\useref{R.DI.HLD}.

\item
Внутренние компоненты и интерфейсы~\useref{R.DI.LLD}
(начиная с уровня 2).

\item
Cредства разработки~\useref{R.DI.Tools}.

\item
Языки программирования~\useref{R.DI.Language}.

\item
Система управления конфигурацией~\useref{R.LC.CMSystem}.

\item
Система поставки~СКЗИ потребителю~\useref{R.LC.Delivery}.

\item
Методы контроля инсталляционных программ~\useref{R.LC.Authenticode}
(начиная с уровня 2).

\item
Список руководств~\useref{R.GD.Admin,R.GD.Roles}.

\item
Типичные ошибки операторов~\useref{R.GD.Misuse}
(начиная с уровня 2).

\item
Программа испытаний~\useref{R.TE.Prg}.

\item
Результаты анализа покрытия тестами~\useref{R.TE.Coverage}.

\item
Результаты анализа глубины тестирования~\useref{R.TE.Deep}
(начиная с уровня 2).
\end{enumerate}
\end{enumerate}

Разделы могут объединяться.

\end{appendix}

