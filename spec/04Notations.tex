\chapter{Оформление требований безопасности}\label{Notations}

Требования безопасности разбиваются на группы.
Группы обозначаются двухбуквенными кодами:
КП, РС, УД и др.
%
Требования внутри группы нумеруются последовательно, 
начиная с~единицы.

Требования безопасности задают разбиение~\TOE на два класса.
%
Для каждого требования в круглых скобках перечисляются 
классы~\TOE, на которые требования распространяются.
Возможные варианты: (1), (2) или (1, 2).

В тексте имеются ссылки на требования. Ссылка [T] означает,
что условия требования~T должны быть использованы в месте ссылки.
Ссылка \{T\} означает, что условия или пояснения в месте ссылки
детализируют~T. 

Формулировки требований безопасности могут включать указания на 
необходимость определения списка методов, списка компонентов, 
списка средств и др.
%
Если не оговорено противное, то результатом определения 
может быть пустой список.

