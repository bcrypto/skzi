\begin{appendix}{А}{рекомендуемое}
{Содержание функциональной спецификации}
\label{SPEC}

\mbox{}

\begin{enumerate}
\item
{Описание~\TOE.}

\begin{enumerate}
\item
Назначение.
\item
Класс ($1$ или $2$).
\item
Основные функциональные возмножности.
\item
Криптографическая граница.
\item
Список критических системных компонентов~\useref{CSCList}.
\end{enumerate}

\item
{Криптографическая поддержка.}

\begin{enumerate}
\item
Список криптографических алгоритмов и протоколов~\useref{CryptoAlg}.

\item
Методы генерации долговременных параметров и ключей~\useref{CryptoGen}.

\item
Средства контроля времени выполнения криптографических алгоритмов 
и протоколов~\useref{CryptoTiming} (только для класса 2).
\end{enumerate}

\item
{Реализация сервисов.}
\begin{enumerate}
\item
Список сервисов~\useref{Services}.

\item
Средства защиты от создания неявных копий
критических объектов~\useref{LeakProtect} 
(только для класса 2).
\end{enumerate}

\item
{Управление доступом.}

\begin{enumerate}
\item
Список ролей операторов~\useref{Roles}.

\item
Список объектов~\useref{Objects}.

\item
Описание политики управления доступом~\useref{DAC}.

\item
Состояния системного сеанса и правила перехода между состояниями~\useref{States}.
\end{enumerate}

\item
{Защита объектов.}

\begin{enumerate}
\item
Криптографические методы обеспечения конфиденциальности~\useref{DPTCryptoEncr}.

\item
Криптографические методы контроля целостности~\useref{DPTCryptoIntegrity}.

\item
Аппаратные методы защиты~\useref{DPTHard}.

\item
Методы разделения секрета~\useref{DPTSplit}.

\item
Алгоритмические методы контроля целостности~\useref{DPTAlgo}.

\item
Организационные методы обеспечения конфиденциальности~\useref{DPTOrg}.

\item
Соответствие <<объекты~--- методы защиты>>~\useref{DPTCrit}, \useref{DPTPublic}.

\item
Методы очистки критических сеансовых объектов~\useref{DPTZeroization}.
\end{enumerate}

\item
{Самотестирование.}

\begin{enumerate}
\item
Проверка работоспособности критических системных компонентов~\useref{CSCTests}.

\item
Перечень проверок самотестирования~\useref{SelfTests}.

\item
Обработка ошибок тестирования~\useref{TestLock}.
\end{enumerate}

\item
{Генерация случайных чисел.}

\begin{enumerate}
\item
Описание генераторов случайных чисел~\useref{RNG}.

\item
Оценка энтропии источников случайности~\useref{Entropy}.

\item
Тестирование выходных последовательностей генераторов~\useref{RNGTests}.
\end{enumerate}

\item
{Идентификация и аутентификация.}

\begin{enumerate}
\item
Методы идентификации операторов~\useref{Identification}.

\item
Методы аутентификации операторов~\useref{AuthData}, \useref{Authentication}.

\item
Проверка качества секретов аутентификации~\useref{PwdSet}.
\end{enumerate}

\item
{Настройка среды.}

\begin{enumerate}
\item
Настройка среды для безопасной установки~\useref{ENVInstall}.

\item
Настройка среды для защиты системных объектов~\useref{ENVObjects}.

\item
Настройка среды для защиты сеансов~\useref{ENVSession}.
\end{enumerate}

\item
Гарантийные меры (могут быть определены в отдельных документах).
\begin{enumerate}
\item
Описание программ~\useref{ProgSpec}.

\item
Внешние интерфейсы программ~\useref{HLD}.

\item
Внутренние компоненты и интерфейсы~\useref{LLD}
(только для класса 2).

\item
Cредства разработки~\useref{Tools}.

\item
Языки программирования~\useref{Language}.

\item
Система управления конфигурацией~\useref{CMSystem}.

\item
Система поставки~\TOE потребителю~\useref{Delivery}.

\item
Методы контроля инсталляционных программ~\useref{Authenticode}
(только для класса 2).

\item
Список руководств~\useref{AdminGuide}, \useref{UserGuide}.

\item
Типичные ошибки операторов~\useref{Misuse}
(только для класса 2).

\item
Программа испытаний~\useref{TestProgram}.

\item
Результаты анализа покрытия тестами~\useref{TestCoverage}.

\item
Результаты анализа глубины тестирования~\useref{TestDeep}
(только для класса 2).
\end{enumerate}
\end{enumerate}

Допускается объединять разделы.
Если в~\TOE не реализован необязательный механизм безопасности, 
то соответствующий раздел можно опустить.

\end{appendix}

