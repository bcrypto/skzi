\begin{appendix}{В}{справочное}
{Памятка по анализу исходных текстов программ}
\label{REVIEW} 

\mbox{}

В программах СКЗИ должны соблюдаться следующие условия.

\begin{enumerate}
\item 
Переменные инициализируются перед использованием.

\item 
Не нарушаются границы массивов.

\item 
Переменные вещественного типа (с плавающей точкой) не используются в операциях
сравнения.

\item 
Динамическая память освобождается.

\item 
Фрагменты памяти (переменные) с критическими данными очищаются перед 
завершением работы с ними. 

\item
Оптимизатор не отключает бесполезную (на его взгяд) очистку.

\item 
Все разобранные варианты условных переходов возможны.

\item 
Все адреса безусловных переходов доступны.

\item 
Каждый цикл завершается за конечное число шагов.

\item 
Нет недостижимых участков кода.

\item 
Цепочки последовательных действий (например, открытие файла, 
чтение из файла, закрытие файла) корректны. 

\item 
При вызове функции соблюдается ее интерфейс.

\item 
Соблюдаются предусловия функций.

\item 
Возвращаемые функциями значения не игнорируются и корректно интерпретируются.

\item 
Исключительные ситуации обрабатываются. 

\item 
Учитываются все возможные типы исключительных ситуаций. 

\item 
Пространство ключей не сужается.

\item 
Используются средства синхронизации (для многозадачных программ).

\item 
Обрабатываются граничные ситуации криптографических алгоритмов 
(например, хэширование пустого сообщения). 

\item 
Обрабатываются исключительные ситуации криптографических протоколов 
(например, обрыв связи). 
\end{enumerate}

\end{appendix}
