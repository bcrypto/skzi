% Environment Setup
\section{Настройка среды (НС)}\label{ES}

\subsection{Обзор}\label{ES.Defs}

Пакет НС устанавливает требования по настройке системной среды СКЗИ.
%
Настройка повышает гарантии безопасности системных объектов СКЗИ,
увеличивает доверие к КСК, усиливает контроль над средой.
%
Настройка особенно важна для программных СКЗИ, которые сильно зависимы от 
среды.

Настройка системной среды выполняется администратором СКЗИ.
Некоторые настройки могут автоматически выполняться в самой среде. 
%
Системная среда или отдельные ее компоненты могут быть сразу же 
настроены надлежащим образом разработчиком СКЗИ или разработчиками компонентов.  
%
В таких случаях для выполнения требований пакета НС достаточно обосновать факт 
безопасной настройки.

Универсальная операционная система может допускать гибкую настройку.
При настройке следует сконфигурировать механизмы 
аутентификации~\useref{R.IA.Auth,R.IA.PwdSet} и
обеспечить~\forref{R.ES.Install,R.ES.Objects}:
\begin{itemize}
\item
соответствие между пользователями (ролями пользователей) 
операционной системы и операторами (ролями операторов) 
СКЗИ~\useref{R.AC.Roles};
\item
запрет на установку программ СКЗИ пользователями операционной системы,
которые не соотнесены с администраторами СКЗИ;
\item
контроль доступа к каталогам и файлам, в которых хранятся 
системные объекты и объекты явных операторов;
\item
соответствие прав доступа к каталогам и файлам 
политик\addendum{е} управления доступом СКЗИ~\useref{R.AC.Policy}.
\end{itemize}

Если операционная система включает подсистему управления процессами,
то следует обеспечить достижение следующих свойств 
(или обосновать их безусловное выполнение)~\forref{R.ES.Session}:
\begin{itemize}
\item
процесс операционной системы, включая его адресное пространство, не может быть 
изменен другими процессами;
\item
адресное пространство процесса не может быть прочитано другими процессами.
\end{itemize}

\subsection{Требования}\label{ES.Reqs}

\req{НС}{1--4}\label{R.ES.Install} % 1
Если предусмотрена установка СКЗИ в системной среде, то должны быть определены
настройки среды для безопасной установки уполномоченным администратором.

\req{НС}{1--4}\label{R.ES.Objects} % 2
Должны быть определены настройки системной среды для защиты долговременных
объектов~\useref{R.AC.Objects} при их хранении внутри криптографической границы.
%
Выбранные настройки должны предотвращать изменение долговременных объектов вне
сеансов между операторами и~СКЗИ.

\req{НС}{1--4}\label{R.ES.Session} % 3
Должны быть определены настройки системной среды для обеспечения
конфиденциальности и целостности сеансовых объектов~\useref{R.AC.Objects} и
аутентификационных данных~\useref{R.IA.AuthData} при их передаче и  обработке в
КСК во время сеансов операторов.

\req{НС}{2--4}\label{R.ES.CVE} % 4
Должны отслеживаться уязвимости КСК. При их обнаружении должны устанавливаться
обновления, осуществляться переход на признаваемые надежными версии КСК или на
другие КСК.

\begin{note*}
Уязвимости могут отслеживаться администратором по аналитическим материалам из
доступных источников или автоматически самой системной средой через обращение к
репозиториям с обновлениями, в которых уязвимости устранены.
%
Отслеживание может проводиться во время разработки СКЗИ. Разработчик использует
результаты отслеживания при выборе надежных КСК или надежных версий КСК.
\end{note*}

\req{НС}{2--4}\label{R.ES.AuthCode} % 5
В системной среде должно быть разрешено устанавливать только те программы и
обновления программного обеспечения КСК, целостность и подлинность которых
подтверждена системной средой.

\req{НС}{3, 4}\label{R.ES.NoCode} % 6
В системной среде должно быть запрещено устанавливать дополнительные программы.
Возможно только обновление программ СКЗИ.

