% Indentification and Authentication
\section{Идентификация и аутентификация (ИА)}\label{IA}

\subsection{Обзор}\label{IA.Intro}

Пакет ИА устанавливает требования по идентификации и аутентификации операторов 
СКЗИ. Аутентификация состоит в проверке принадлежности оператора к определенной 
роли и допустимости выполнения сервисов данной роли. Идентификация поддерживает 
аутентификацию, а также управление доступом: идентификатор оператор 
учитывается при проверке прав доступа.
%
Средства идентификации и аутентификации могут быть полностью или частично 
реализованы самим~СКЗИ.

Аутентификация выполняется в начале сеанса оператора.
Допускается, что для выполнения некоторых сервисов 
или доступа к некоторым объектам требуется дополнительная 
аутентификация~\forref{R.IA.Auth}. 

Методы аутентификации основываются на использовании комбинаций из трех факторов:
\begin{itemize}
\item
владение устройствами аутентификации (<<что я имею>>), 
\item
знание секретов аутентификации (<<что я знаю>>), 
\item
обладание биометрическими характеристиками (<<кто я>>).
\end{itemize}

Рекомендуется задействовать более одного фактора, например использовать 
карты доступа вместе с PIN-кодами доступа~\forref{R.IA.2FA}.

При аутентификации операторы предъявляют устройства аутентификации 
(смарт-карты) и аутентификационные данные (пароли, биометрика).
%
Средства аутентификации проверяют корректность представленных устройств и
сравнивают характеристики аутентификационных данных с контрольными 
значениями~\forref{R.IA.AuthData}.

При создании и изменении секретов аутентификации предусматривается проверка их
качества. Проверка может состоять в контроле длины пароля или контроле включения
в пароль как цифровых, так и буквенных символов в различных регистрах. Проверка
качества секретов не может основываться на ограничениях в руководствах, т.~е. на
предположении о том, что оператор обязательно задаст пароль нужного
качества~\forref{R.IA.PwdSet}.

\subsection{Требования}\label{IA.Reqs}

\req{ИА}{1--4}\label{R.IA.Id}
Каждому оператору должен быть назначен идентификатор и набор 
ролей~\useref{R.AC.Roles}.

\req{ИА}{1--4}\label{R.IA.AuthData}
Для каждого идентификатора оператора~\useref{R.IA.Id}
должны быть определены аутентификационные данные. 
Среди аутентификационных данных должны быть выделены 
секреты аутентификации.
%
Устройства ввода аутентификационных данных должны быть~\forref{R.ST.CSCList} 
включены в список критических системных компонентов. 

\req{ИА}{1--4}\label{R.IA.Auth}
Должны быть определены и корректно реализованы средства 
аутентификации для проверки подлинности оператора и возможности 
выполнения оператором сервисов явных ролей~\useref{R.AC.Policy}.
%
Средства аутентификации, реализуемые~СКЗИ, 
должны быть~\forref{R.SV.List} включены в список сервисов.

\req{ИА}{2--4}\label{R.IA.2FA}
Должно использоваться не менее двух факторов аутентификации.

\begin{note}
Двухфакторную аутентификацию обеспечивает аппаратный токен (<<что я имею>>), 
для активации которого нужно ввести  PIN (<<что я знаю>>).
\end{note}

\req{ИА}{1--4}\label{R.IA.AuthStrength}
Вероятность пройти аутентификацию, 
не зная секретов аутентификации, 
не должна превышать~$10^{-6}$, 
если предпринимается одна попытка аутентификации, 
и не должна превышать $10^{-5}$, 
если предпринимаются попытки в течение $1$~мин.

\begin{note}
Требование будет выполнено, если в качестве секрета используется случайный PIN
из $6$~десятичных символов, и если после каждой неверной попытки аутентификации
выполняется $6$-секундная задержка.
\end{note}

\req{ИА}{1--4}\label{R.IA.PwdMask}
При вводе секрета аутентификации информация о нем может отображаться только на 
короткое время для контроля корректности ввода.

\begin{note}
Информация о вводимом секрете обычно маскируется, например, символы пароля
представляются звездочками. Для контроля корректности звездочки могут появляться
через доли секунд после ввода очередного символа, или весь введенный пароль может
быть отображен при удержании оператором определенной кнопки графического
интерфейса.
\end{note}

\req{ИА}{1--4}\label{R.IA.PwdSet}
Должны быть определены и реализованы средства проверки качества секретов
аутентификации. Средства должны применяться при каждой установке или смене
секрета.

\begin{note}
При установке и смене секрета оператору может предлагаться использовать
уже сгенерированный секрет нужного качества.
\end{note}

\req{ИА}{1--4}\label{R.IA.AuthProtect}
При реализации средств аутентификации в~СКЗИ контрольные значения
аутентификационных данных должны быть~\forref{R.AC.Objects} отнесены к открытым
или критическим объектам.
%
Контрольное значение должно быть отнесено к критическому объекту, 
если оно касается секрета аутентификации и по нему за приемлемое
время можно определить секрет.
%
Сеансовые объекты, которые содержат значения секретов аутентификации,
должны быть~\forref{R.AC.Objects} отнесены к критическим объектам.

