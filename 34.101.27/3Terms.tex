\chapter{Термины и определения}\label{Terms}

В настоящем стандарте применяют  
%
%термины в соответствии с СТБ 1176.1-99, СТБ 1176.2-99,
%СТБ~34.101.1~--- СТБ~34.101.3, 
%СТБ 1221, СТБ~34.101.31, ГОСТ 28147-89,
%а также 
%
следующие термины с соответствующими определениями:

{\bf \thedefctr~аутентификация:}
Проверка подлинности идентификатора оператора.

%Authenticate: To confirm the identity of an entity when that identity is presented.

{\bf \thedefctr~аутентификационные данные:}
Информация, которая используется для аутентификации.

%Information used to verify the claimed identity of a user

{\bf \thedefctr~генератор случайных чисел:}
Аппаратно-программное устройство, 
которое вырабатывает последовательность непредсказуемых 
элементов.

%aппаратно-программное устройство, 
%вырабатывающее последовательность чисел, 
%каждый следующий элемент которой трудно
%предсказать по всем предыдущим элементам.

{\bf \thedefctr~долговременный объект:}
Объект, который хранится в пределах криптографической границы или передается
за ее пределы, операции над которым могут выполняться в нескольких сеансах.

{\bf \thedefctr~зашифрование}:
Преобразование объектов,
направленное на обеспечение их конфиденциальности,
которое осуществляется с использованием секретного или открытого ключа.

% belt: Преобразование сообщения,
% направленное на обеспечение его конфиденциальности,
% которое определяется с использованием ключа.

{\bf \thedefctr~защита объектов:}
Контроль целостности и обеспечение конфиденциальности критических объектов,
контроль целостности открытых объектов.

{\bf \thedefctr~идентификация:}
Присвоение операторам уникальных идентификаторов.

{\bf \thedefctr~имитовставка}:
Контрольная характеристика объекта, 
которая определяется с использованием секретного ключа 
и служит для контроля целостности и подлинности объекта.

% belt: Двоичное слово, 
% которое определяется по сообщению с использованием ключа 
% и служит для контроля целостности и подлинности сообщения.

{\bf \thedefctr~имитозащита}:
Контроль целостности объектов, 
который реализуется путем выработки и проверки имитовставок.

% old: Выработка или проверка имитовставок.

% belt, brng: Контроль целостности сообщений, 
% который реализуется путем выработки и проверки имитовставок.

{\bf \thedefctr~клиентская программа:}
Программа, которая от лица оператора вызывает сервисы
программного средства криптографической защиты информации.

{\bf \thedefctr~конфиденциальность}:
Гарантия того, что объекты доступны для использования
только тем сторонам, которым они предназначены.

{\bf \thedefctr~криптографическая граница:} 
Точно определенный разработчиком непрерывный физический 
периметр в среде эксплуатации, 
который определяет контролируемую границу программного средства 
криптографической защиты информации.

{\bf \thedefctr~криптографический алгоритм:}
Алгоритм преобразования объектов,
направленный на обеспечение их конфиденциальности,
контроля целостности или подлинности,
в том числе алгоритм управления криптографическими
ключами для защиты объектов.

{\bf \thedefctr~криптографический ключ:} 
Объект-параметр, используемый вместе с криптографическим алгоритмом 
для управления операциями зашифрования и расшифрования,
вычисления и проверки электронной цифровой подписи,
выработки и проверки имитовставки,
выработки псевдослучайных данных,
выработки совместно используемой конфиденциальной информации.

% bign: Параметр, который управляет криптографическими 
% операциями зашифрования и расшифрования, 
% выработки и проверки ЭЦП, 
% генерации псевдослучайных чисел и др.

{\bf \thedefctr~криптографический протокол:}
Точно определенные последовательность действий или набор правил, 
предусматривающие взаимодействие двух и более сторон 
с использованием криптографических алгоритмов.

{\bf \thedefctr~критические системные компоненты:}
Находящееся внутри криптографической границы 
аппаратное и программное обеспечение, которое используется
для передачи, обработки и хранения объектов
программного средства криптографической защиты информации.

{\bf \thedefctr~критический объект:} 
Объект, несанкционированные раскрытие или модификация которого 
снижают безопасность.

{\bf \thedefctr~личный ключ:}
Криптографический ключ, используемый
вместе с криптографическим алгоритмом с открытым ключом, 
который однозначно связан с конкретным оператором 
и не является общедоступным.

% bign: Ключ, который связан с конкретной стороной, не является общедоступным
% и используется в настоящем предстандарте для выработки ЭЦП 
% и для разбора токена ключа.

{\bf \thedefctr~неявная копия:}
Копия объекта, переданная по побочному каналу.

{\bf \thedefctr~объект}: 
Элемент, который содержит или получает информацию
и над которым выполняются операции.

{\bf \thedefctr~оператор:}
Лицо или клиентская программа, выступающая от имени лица, 
которые взаимодействуют с программным средством криптографической защиты 
информации.

{\bf \thedefctr~открытый объект:} 
Объект, несанкционированная модификация 
которого снижает безопасность, 
а раскрытие~--- не снижает.

{\bf \thedefctr~открытый ключ:}
Криптографический ключ, используемый 
вместе с криптографическим алгоритмом с открытым ключом, 
который строится по личному ключу  
и может быть сделан общедоступным.

% bign: Ключ, который строится по личному ключу, 
% связан с конкретной стороной, 
% может быть сделан общедоступным
% и используется в настоящем предстандарте для проверки ЭЦП 
% и для создания токена ключа.

{\bf \thedefctr~побочный канал}:
Нежелательный дополнительный канал передачи информации о 
входных, промежуточных или выходных данных 
криптографического алгоритма или протокола, 
возникающий из-за особенностей его 
аппаратно-программной реализации.

{\bf \thedefctr~подлинность}:
Гарантия того, что сторона действительно 
является владельцем (создателем, отправителем) 
определенного объекта.

{\bf \thedefctr~политика управления доступом}:
Правила, определяющие допустимые операции 
операторов над сервисами и объектами.

{\bf \thedefctr~программное средство криптографической защиты информации;
\TOE}:
Средство криптографической защиты информации, выполненное целиком 
программно, без аппаратных компонентов.

{\bf \thedefctr~разделение секрета}:
Разбиение критического объекта на частичные секреты, 
каждый из которых по отдельности или даже вместе с некоторыми
другими частичными секретами не дает информации об исходном объекте.

{\bf \thedefctr~расшифрование}:
Преобразование, обратное зашифрованию, которое определяется с помощью
секретного или личного ключа.

% belt: Преобразование, обратное зашифрованию.

{\bf \thedefctr~сеанс оператора:}
Период взаимодействия оператора с программным средством криптографической
защиты информации.

{\bf \thedefctr~сеансовый объект:}
Объект, который создается, 
используется и уничтожается в течение одного сеанса.

{\bf \thedefctr~секрет аутентификации:}
Пароль, PIN-код и другие аутентификационные данные, 
которые однозначно связаны с конкретным оператором 
и не являются общедоступными.

{\bf \thedefctr~секретный ключ:}
Криптографический ключ, используемый
вместе с криптографическим алгоритмом с секретным ключом, 
который однозначно связан с конкретным оператором 
или группой операторов и не является общедоступным.

% belt: Параметр, который управляет операциями шифрования 
% и имитозащиты и который известен только определенным сторонам.

{\bf \thedefctr~сервис:}
Реализованная в программном средстве криптографической защиты информации 
и доступная оператору функция.

{\bf \thedefctr~синхропосылка:}
Открытые входные данные криптографического алгоритма или протокола,
которые обеспечивают уникальность результатов 
криптографического преобразования на фиксированном ключе.

% belt: Открытые входные данные криптографического алгоритма,
% которые обеспечивают уникальность результатов 
% криптографического преобразования на фиксированном ключе.

{\bf \thedefctr~системный сеанс:}
Непрерывный период работы программного средства криптографической 
защиты информации.

{\bf \thedefctr~среда экcплуатации}:
Аппаратно-программное обеспечение, 
организационные процедуры и мероприятия, 
необходимые для функционирования 
программного средства криптографической защиты информации.

{\bf \thedefctr~средство криптографической защиты информации;
СКЗИ}:
Набор аппа\-ратно-программных компонентов, 
который реализует криптографические алгоритмы и протоколы,
а также возможно дополнительные средства управления ключами,
контроля доступа и проверки работоспособности, 
предназначенные для безопасного управления вызовами
криптографических алгоритмов и их входными и выходными данными.

{\bf \thedefctr~целостность}:
Гарантия того, что объект не изменен при хранении или передаче.

{\bf \thedefctr~шифрование}:
Зашифрование или расшифрование.

{\bf \thedefctr~хэш-значение}:
Контрольная характеристика объекта, 
которая определяется без использования ключа и 
служит для контроля целостности объекта и для представления 
объекта в сжатой форме.

% belt, bign: Двоичное слово фиксированной длины, 
% которое определяется по сообщению без использования ключа и 
% служит для контроля целостности сообщения и для представления 
% сообщения в сжатой форме.

{\bf \thedefctr~хэширование}:
Выработка хэш-значений.

% belt: +

{\bf \thedefctr~частичный секрет}:
Критический объект, 
полученный в результате применения метода разделения секрета.

% bels: конфиденциальные данные пользователя, используемые для 
% восстановления секрета

{\bf \thedefctr~электронная цифровая подпись; ЭЦП}:
Контрольная характеристика объекта, 
которая определяется с использованием личного ключа, 
проверяется с использованием открытого ключа,
служит для контроля целостности и подлинности объекта 
и обеспечивает невозможность отказа от авторства.

% bign: Двоичное слово, которое служит для контроля 
% целостности и подлинности сообщения, 
% обеспечивает невозможность отказа от авторства,
% определяется с использованием личного ключа и проверяется 
% с использованием открытого ключа.



