\chapter{Функциональные требования безопасности к программному средству 
криптографической защиты информации}\label{FReqsTOE}

\section{Требования по криптографической поддержке (КП)}

\req{КП}{1, 2}\label{CryptoAlg}
Должны быть определены и корректно реализованы криптографические алгоритмы 
и протоколы~\TOE. 
%
Каждый алгоритм и протокол должен быть однозначно идентифицирован:
должен быть указан его тип, 
дана ссылка на спецификацию, 
определены режим работы, 
поддерживаемые длины ключей.

\req{КП}{1, 2}\label{CryptoStd}
Спецификации криптографических алгоритмов и протоколов~\useref{CryptoAlg}
должны быть приняты в качестве технических нормативно-правовых актов 
(далее~--- ТНПА).

\req{КП}{1, 2}\label{CryptoGen}
Если в~\TOE предусмотрены методы генерации долговременных параметров и ключей
криптографических алгоритмов и протоколов~\useref{CryptoAlg}, 
то эти методы должны быть определены и корректно реализованы.
%
Методы генерации должны соответствовать спецификациям
на алгоритмы и протоколы или уточнять данные спецификации.

\req{КП}{2}\label{CryptoTiming}
Криптографические алгоритмы и протоколы~\useref{CryptoAlg},
а также методы генерации их ключей,
должны быть реализованы так,
чтобы по времени их выполнения нельзя было сделать вывод об
используемых или генерируемых личных и секретных ключах.

\section{Требования по реализации сервисов (РС)}

\req{РС}{1, 2}\label{Services}
Должны быть определены и корректно реализованы сервисы~\TOE.
%
Для каждого сервиса должно быть определено его назначение, 
входные и выходные данные. 
%
Если сервисы выполняются в определенной последовательности,
то данная последовательность должна быть определена.

\req{РС}{1, 2}\label{ObligServices}
В список сервисов должны быть включены~\forref{Services}:
\begin{itemize}
\item[--]
сервис вывода номера версии~\TOE;
\item[--]
сервисы самотестирования~\useref{CSCTests}, \useref{SelfTests};
\item[--]
по крайней мере один сервис,
который реализует криптографический алгоритм или протокол~\useref{CryptoAlg}.
\end{itemize}

%\req{РС}{2}\label{Emergency}
%Должны быть реализованы сервисы принудительной очистки
%всех доступных критических объектов~\useref{CryptoAlg}.

\req{РС}{1, 2}\label{InputProtect}
На любых входных данных сервисы~\useref{Services}
должны нормально завершаться, \ie возвращать правильные результаты, 
в том числе признаки некорректных входных данных.

\req{РС}{2}\label{LeakProtect}
Сервисы~\useref{Services}, 
которые выполняют операции над критическими объектами~\useref{Objects}, 
должны быть реализованы так, чтобы после их завершения 
в пределах криптографической границы
не оставалось неявных копий критических объектов в открытом виде.

\section{Требования по управлению доступом (УД)}

\req{УД}{1, 2}\label{Roles}
Должны быть определены роли операторов~\TOE.
Должна быть предусмотрена роль <<Администраторы>>. 

\req{УД}{1, 2}\label{Objects}
Должны быть определены объекты~\TOE.
Для каждого объекта должно быть задано его назначение,
проведена классификация 
(открытый или критический, сеансовый или долговременный),
определен владелец.

\req{УД}{1, 2}\label{DAC}
Должна быть определена и корректно реализована 
политика управления доступом~\TOE.
Политика должна устанавливать набор допустимых 
операций операторов различных ролей~\useref{Roles} 
над сервисами~\useref{Services}  и сервисов, 
выступающих от имени операторов, над объектами~\useref{Objects} 
и другими сервисами.

\req{УД}{1, 2}\label{States}
Должны быть определены состояния системного сеанса~\TOE.
Должны быть определены и корректно реализованы правила перехода между состояниями.
Должны быть предусмотрены состояния, соответствующие 
сеансам операторов различных ролей~\useref{Roles}, и состояние блокировки.

\req{УД}{1, 2}\label{OpStates}
В состояниях, соответствующих сеансам операторов различных ролей,
должна действовать политика управления доступом~\useref{DAC}
относительно данных ролей.
Перед переходом в состояния должна проводиться аутентификация
операторов~\useref{Authentication}.

\req{УД}{1, 2}\label{LockState}
В состоянии блокировки должно быть запрещено выполнение всех
сервисов, кроме сервисов самотестирования~\useref{SelfTests},
сервисов аутентификации администратора~\useref{Authentication} 
и вспомогательных сервисов, не связанных 
с обработкой объектов внутри криптографической границы.
%
При переходе в состояние блокировки должны быть завершены 
все открытые сеансы операторов. 
Если системный сеанс~\TOE завершен в состоянии блокировки,
то и начаться он должен в этом состоянии.

\section{Требования по защите объектов (ЗО)}

\req{ЗО}{1, 2}\label{DPTCrit}
Должна обеспечиваться конфиденциальность критических объектов~\useref{Objects}.
Для этого должны использоваться криптографические методы, 
аппаратные методы или методы разделения секрета. 
Для обеспечения конфиденциальности частичных секретов
могут дополнительно использоваться организационные меры.

\req{ЗО}{1, 2}\label{DPTPublic}
Должен осуществляться контроль целостности критических и открытых 
объектов~\useref{DAC}.
Для этого должны использоваться криптографические, аппаратные 
или алгоритмические методы. 
%
%Для контроля целостности системных объектов и частичных секретов
%могут дополнительно использоваться алгоритмические методы.

\req{ЗО}{1, 2}\label{DPTCryptoEncr}
Должны быть определены и корректно реализованы
криптографические методы обеспечения конфиденциальности.
Методы должны быть основаны на алгоритмах шифрования~\useref{CryptoAlg}.
%
Личные и секретные ключи алгоритмов должны быть отнесены к критическим объектам, 
а открытые ключи и долговременные параметры~--- 
к открытым объектам~\forref{Objects}.
%
%Ключ шифрования не должен быть слабее защищаемых критических объектов.

\req{ЗО}{1, 2}\label{DPTCryptoIntegrity}
Должны быть определены и корректно реализованы
криптографические методы контроля целостности.
Методы должны быть основаны на алгоритмах ЭЦП и
имитозащиты~\useref{CryptoAlg}.
%
Личный ключ ЭЦП и секретный ключ имитозащиты должны быть отнесены 
к критическим объектам, а открытый ключ ЭЦП и долговременные параметры~---
к открытым объектам~\forref{Objects}.
%
Длина имитовставки должна выбираться так, 
чтобы вероятность необнаружения 
модификации объекта нарушителем, который не знает ключ, 
не превышала~$2^{-32}$.

\req{ЗО}{1, 2}\label{DPTHard}
Должны быть определены и корректно использованы аппаратные методы защиты.
Устройства, которые реализуют аппаратные методы, 
должны соответствовать ТНПА
в части физической безопасности.

\req{ЗО}{1, 2}\label{DPTSplit}
Должны быть определены и корректно реализованы методы разделения секрета.
При восстановлении критического объекта должно использоваться 
не менее двух различных частичных секретов.
%
Если для восстановления критического объекта требуется $k$ частичных секретов,
то любые $k-1$ частичных секретов не 
должны давать никакой информации об исходном объекте.
%
Частичные секреты должны быть отнесены к критическим объектам~\forref{Objects}.
%
Владельцы частичных секретов должны быть различны~\forref{Objects}.

\req{ЗО}{1, 2}\label{DPTAlgo}
Должны быть определены и корректно реализованы алгоритмические методы 
контроля целостности. 
Алгоритмические методы контроля должны гарантировать, 
что вероятность необнаружения случайной модификации контролируемого
объекта не превышает~$2^{-32}$. 
Если контролируемый объект не является частичным секретом или системным объектом, 
то его контрольная характеристика 
должна быть отнесена к открытым или критическим объектам~\forref{Objects}.

\req{ЗО}{1, 2}\label{DPTOrg}
Должны быть определены и изложены в руководствах~\forref{AdminGuide}, \forref{UserGuide}
организационные меры по обеспечению конфиденциальности частичных секретов.
Меры должны быть направлены на ограничение физического 
доступа к носителям информации, на которых хранятся секреты, 
и попаданию порогового числа частичных секретов в руки одного лица.

\req{ЗО}{1, 2}\label{DPTCritApply}
При экспорте критических объектов
должна устанавливаться их защита~\useref{DPTCrit}.

\req{ЗО}{1, 2}\label{DPTPublicApply}
При импорте критических и открытых объектов
должен проводиться контроль их целостности~\useref{DPTPublic}.
При нарушении целостности использование объекта должно быть запрещено.

\req{ЗО}{1, 2}\label{DPTSystemApply}
Контроль целостности системных объектов~\useref{DPTPublic} 
должен проводиться при самотестировании~\forref{SelfTests}.

\req{ЗО}{1, 2}\label{DPTZeroization}
Все сенсовые критические объекты~\useref{Objects}
должны очищаться до завершения сеансов.

\section{Требования по самотестированию (СТ)}

\req{СТ}{1, 2}\label{CSCList}
Должны быть определены критические системные компоненты~\TOE.

\req{СТ}{1, 2}\label{CSCTests}
При начальном запуске~\TOE должно проверять состав 
и работоспособность критических системных компонентов~\useref{CSCList}.

\req{СТ}{1, 2}\label{SelfTests}
При начальном запуске, а также по запросу оператора 
должно проводиться тестирование, 
обязательно включающее:
\begin{itemize}
\item[--]
тесты криптографических алгоритмов и протоколов~\useref{CryptoAlg};
\item[--]
контроль целостности всех системных объектов, включая файлы 
программ~\useref{DPTSystemApply};
\item[--]
тесты для генераторов случайных чисел~\useref{RNGTests}.
\end{itemize}

\req{СТ}{1, 2}\label{TestData}
Тестовые данные криптографических алгоритмов и протоколов 
(ключи, открытые тексты, шифртексты) 
должны быть отнесены к открытым объектам~\forref{Objects}.

\req{СТ}{1, 2}\label{TestLock}
При ошибках тестирования системный сеанс~\TOE должен переходить в состояние
блокировки~\forref{LockState}.

\section{Требования по генерации случайных чисел (СЧ)}

\req{СЧ}{1, 2}\label{RNG}
Должны быть определены генераторы случайных чисел, 
которые используются для выработки ключей и других критических 
объектов~\useref{CryptoGen}. 
%
Для каждого генератора должны быть указаны 
источники случайности и методы обработки данных от источников случайности.
%
Генераторы, которые не являются компонентами~\TOE, 
должны быть включены в список 
критических системных компонентов~\useref{CSCList}.

\req{СЧ}{1, 2}\label{Entropy}
Для каждого генератора случайных чисел~\useref{RNG}
должна быть проведена оценка энтропии всех его источников случайности. 
Способ обработки данных от источников случайности должен гарантировать, 
что собранные данные содержат достаточно неопределенности 
для надежной генерации критического объекта.

\req{СЧ}{1}\label{RndSource1}
Если в генераторе случайных чисел~\useref{RNG} отсутствуют физические 
источники случайности, то должно использоваться не менее 
двух альтернативных разнотипных источников.

\req{СЧ}{2}\label{RndSource2}
Каждый генератор случайных чисел~\useref{RNG} должен обязательно
использовать хотя бы один физический источник случайности.

\req{СЧ}{1, 2}\label{RNGTests}
Должна быть разработана и корректно реализована процедура 
тестирования выходных последовательностей генератора 
случайных чисел~\useref{RNG},
в котором используются физические источники случайности.
%
Процедура должна быть направлена на выявление 
отказов, сбоев и изменений физических параметров
при функционировании физических источников.

\req{СЧ}{1, 2}\label{RNGHash}
Выходные данные генератора случайных чисел~\useref{RNG}
должны являться результатом применения 
криптографических алгоритмов~\useref{CryptoAlg}
к данным от источников случайности, 
возможно дополненным обновляемым внутренним состоянием
или предыдущими случайными числами.
%
Применяемые криптографические алгоритмы должны 
обеспечивать сложные зависимости между выходными данными генератора и 
данными от каждого из источников случайности.
%
%\addendum{Криптографические алгоритмы должны применяться так, 
%чтобы каждый бит выходного случайного числа нелинейно 
%зависел от любого бита данных от каждого из источников случайности.}

