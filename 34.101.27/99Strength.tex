\chapter{Приложение A. Уровни стойкости}

Типовые задачи и значения параметров безопасности для надежных алгоритмов
представлены в таблице~\ref{Table.COMMON.Strength}.

\begin{table}[bht]
\caption{Параметры безопасности}\label{Table.COMMON.Strength}
\begin{center}
\begin{tabular}{|p{7.3cm}|p{6cm}|p{2.5cm}|}
\hline
Задача & Размерность, $l$ & Параметр безопасности\\
\hline
\hline
определение ключа симметричного алгоритма & 
длина секретного ключа в битах & $l$\\
%
\hline
обращение функции хэширования & 
длина хэш-значения в битах & $l$\\
%
\hline
построение коллизии для функции хэширования & 
длина хэш-значения в битах & $l/2$\\
%
\hline
факторизация составного числа $n$ & 
длина $n$ в битах & 
$11.6\sqrt[3]{l}-28.6$\\
%
\hline
дискретное логарифмирование в группе порядка $q$ & 
длина $q$ в битах & 
$l/2$\\
%
\hline
дискретное логарифмирование в подгруппе мультипликативной группы 
конечного поля из $p$ элементов & 
длина $p$ в битах & 
$11.6\sqrt[3]{l}-28.6$\\
\hline
\end{tabular}
\end{center}
\end{table}

